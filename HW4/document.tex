\documentclass[12pt]{article}
\usepackage{amsmath, amssymb, amsfonts, amsthm}
\usepackage{euler}
\title{Abstract Algebra: Homework \#4}
\author{Joel Savitz}
\date{Wednesday 17 June 2020}

\newcommand{\reals}{\mathbb{R}}
\newcommand{\ints}{\mathbb{Z}}
\newcommand{\gltwo}{GL_2(\reals)}
\newcommand{\glmatrix}[4]{\ensuremath{\begin{bmatrix} #1 & #2 \\ #3 & #4 \end{bmatrix}}}
\newcommand{\glxmatrix}[9]{\ensuremath{\begin{bmatrix} #1 & #2 & #3 \\
#4 & #5 & #6 \\ #7 & #8 & #9 \end{bmatrix}}}
\newcommand{\glinverse}[4]{\ensuremath{\frac{1}{#1 #4 - #2 #3}\glmatrix{#4}{-#2}{-#3}{#1}}}
\newcommand{\ord}{\operatorname{ord}}
\newtheorem{thm}{Theorem}
\newtheorem{cnt}{Counterexample}

\begin{document}
\maketitle


\section{Chapter 9, Exercise A3}

Suppose $G_1, G_2,$ and $G_3$ are groups
and let $f:G_1 \to G_2$ and $g:G_2 \to G_3$ be isomorphisms.

\begin{thm} \label{thm1}
	$g \circ f: G_1 \to G_3$ is an isomophism $ \iff G_1 \cong G_3$
\end{thm}

\begin{proof}
	Suppose $a,b \in G_1$.
	Then, since $f$ is an isomorphism from $G_1$ to $G_2$,
	we have $f(ab) = f(a)f(b)$.
	Then, since $g$ is an isomorphism grom $G_2$ to $G_3$,
	we have $g(f(ab)) = g(f(a))g(f(b))$ since
	$f(ab),f(a),f(b) \in G_2$.
	Since $g \circ f$ is defined as $g(f(x))$
	for any $x \in G_1$, we have:
	\begin{align} \label{eq1}
	(g \circ f)(ab) = \big((g \circ f)(a)\big)\big((g \circ f)(b)\big)
	\end{align}
	Since the composition of two bijective functions is bijective
	in general, and since $f$ and $g$ are bijections,
	we have that $(g \circ f)$ is a bijeciton.
	This fact, along with the fact that equation \ref{eq1} holds,
	is exactly the criteria we need to conclude that
	$(g \circ f)$ is an isomorphism from $G_1$ to $G_3$,
	or in other words, $G_1 \cong G_3$.
	Not only does this prove theorem \ref{thm1},
	but this demonstrates the transitivity property
	of the isomorphism relation in general.
\end{proof}

\section{Chapter 9, Exercise B3}

Suppose that $G_1$ and $G_2$ are groups
and $f:G_1 \to G_2$ is an isomorphism.

\begin{thm} \label{thm2}
	If $G_1$ is a cyclic group with a generator element $a \in G_1$
	then $G_2$ is a cyclic group with a generator element $f(a) \in G_2$.
\end{thm}

\begin{proof}
	Assume that $G_2$ is a cyclic group generated by $a \in G_1$.
	Let $n = |G_1|$.
	Since $G_1$ is generated by $a$,
	we can write $G_1$ as:
	\begin{align} \label{eq2}
		G_1 = \bigcup\limits_{i = 1}^n a^i
	\end{align}
	We will show by induction on $i$ that $f(a)$ generates
	every element in $G_2$.
	First, consider the base case of $i = 1$.
	Since $f(a)^1 = f(a) \in G_2$, we have a subset of $G_2$
	that we can write as follows:
	\begin{align} \label{eq3}
		\bigcup\limits_{i = 1}^1 f(a)^i = \{ f(a) \} \subseteq G_2
	\end{align}
	Now we make the induction hypothesis that:
	\begin{align} \label{eq4}
		\bigcup\limits_{i = 1}^m f(a)^i \subseteq G_2
	\end{align}
	holds for for some natural $m \le n$.

	Then, if we take the union of both sides of proposition \ref{eq4}
	with $f(a)^{m+1}$, we have:
	\begin{align} \label{eq5}
		\bigcup\limits_{i = 1}^m f(a)^i \cup f(a)^{m+1} &
		\subseteq G_2 \cup f(a)^{m+1} \\
	\end{align}
	Which simplifies to:
	\begin{align} \label{eq6}
		\bigcup\limits_{i = 1}^{m+1} f(a)^i&
		\subseteq G_2
	\end{align}
	Since $f(a) \in G_2$ and $\Big(f(a)^m \in G_2
	\land f(a)^{m+1} = f(a)^m f(a)\Big) \implies f(a)^{m+1} \in G_2$
	by the definition of $f$.

	By strong induction on $i$,
	we have that proposition \ref{eq6} holds for any value of $m \le n$
	so it must hold for $n$ as well.

	Then we have:
	\begin{align} \label{eq7}
		\bigcup\limits_{i = 1}^{n} f(a)^i&
		\subseteq G_2
	\end{align}

	However, since the set $\bigcup\limits_{i = 1}^{n} f(a)^i$
	has the same cardinality as $G_2$, we must have that
	any element in $G_2$ must be an element of  
	$\bigcup\limits_{i = 1}^{n} f(a)^i$.
	Then, by the axiom of extensionality:
	\begin{align} \label{eq8}
		\Big(G_2 \subseteq
		\bigcup\limits_{i = 1}^{n} f(a)^i \Big)
		\land
		\Big(\bigcup\limits_{i = 1}^n f(a)^i
		\subseteq G_2\Big)
		\iff 
		\bigcup\limits_{i = 1}^n f(a)^i
		= G_2
	\end{align}
	We can rephrase proposition \ref{eq8} by
	saying that $f(a)$ generates $G_2$.

	Finally, since $G_2$ is generated by a single
	element $f(a)$, we have that $G_2$ is a cyclic
	group generated by $f(a)$.
	This proves theorem \ref{thm2}.
\end{proof}

\section{Chapter 9, Exercise E1}

Suppose $E = \{ x \in \ints\ |\ \exists k \in \ints \ni x = 2k \}$.

\begin{thm} \label{thm3}
	$\ints \cong E$ with respect to addition.
\end{thm}

\begin{proof}
	Let $f:\ints \to E$ such that $f(x) = 2x$.
	Let $a,b \in \ints$.
	We then have $f(a) = 2a$ and $f(b) = 2b$.
	Notice that:
	\begin{align}
		2(a + b) & = 2a + 2b \\
		\label{eq9}
		f(a + b) & = f(a) + f(b)
	\end{align}

	Suppose that we have some $c \in \ints$
	where $f(c) = f(a)$.
	Then, $2c = 2a \iff c = a$.
	Thus $f$ is an injection from $\ints$ to $E$.

	Suppose we have some $d \in E$.
	Then, by the definition of $E$
	there exists some integral $p$
	where $d = 2p$,
	so we have $f(p) = 2p = d$.
	Thus $f$ is a surjection from $\ints$ to $E$.

	Then, since $f$ is
	an injection and a surjection,
	we have that $f$ is a bijection.

	Since $f$ is a bijection from
	$\ints$ to $E$ where equation \ref{eq9} holds,
	we have that $f$ is an
	isomorphism from $\ints$ to $E$
	with respect to addition.

	Then since there exists an isomorphism from
	$\ints$ to $E$ with respect to addition,
	we have that $\ints$ is isomorphic to $E$
	with respect to addition,
	i.e. $\ints \cong E$.
	This proves theorem \ref{thm3}.
\end{proof}

\section{Chapter 9, Exercise F2} \label{permsect}

Suppose $G = S_3$ and $G' = \{e, a, b, ab, aba, abab \}$
where $*$ is an operation on $G'$ and we have
$a^2 = e \land b^2 = e \land bab = aba$.

For the group $S_3$,
denote the rotations by $r_1, r_2, r_3$ for a rotation $r_n$ of $\frac{2 n\pi}{3}$ radians,
and denote the flips about each axis of symmetry by $f_1, f_2, f_3$.

More precisely, denote the above permutations as follows:
\begin{align}
	\label{topperm}
	r_1 = & \binom{1\ 2\ 3}{2\ 3\ 1} &
	r_2 = & \binom{1\ 2\ 3}{3\ 1\ 2} &
	r_3 = & \binom{1\ 2\ 3}{1\ 2\ 3} \\
	\label{botperm}
	f_1 = & \binom{1\ 2\ 3}{1\ 3\ 2} &
	f_2 = & \binom{1\ 2\ 3}{3\ 2\ 1} &
	f_3 = & \binom{1\ 2\ 3}{2\ 1\ 3}
\end{align}


Then, table \ref{t1} describes the behavior of $G$ under function composition.

\begin{table}[!ht] 
\begin{tabular}{l|llllll}
	$\circ$ & $r_3$ & $r_1$ & $r_2$ & $f_1$ & $f_2$ & $f_3$	\\ \hline
	$r_3$ & $r_3$ & $r_1$ & $r_2$ & $f_1$ & $f_2$ & $f_3$   \\
	$r_1$ & $r_1$ & $r_2$ & $r_3$ & $f_3$ & $f_1$ & $f_2$ 	\\
	$r_2$ & $r_2$ & $r_3$ & $r_1$ & $f_2$ & $f_3$ & $f_1$ 	\\
	$f_1$ & $f_1$ & $f_3$ & $f_2$ & $r_3$ & $r_2$ & $r_1$ 	\\
	$f_2$ & $f_2$ & $f_1$ & $f_3$ & $r_1$ & $r_3$ & $r_2$ 	\\
	$f_3$ & $f_3$ & $f_2$ & $f_1$ & $r_2$ & $r_1$ & $r_3$	\\
\end{tabular}
\centering
\caption{Operation table for $G$ under $\circ$}
\label{t1}
\end{table}

By the above defining equations, table \ref{t2} is the operation table for
the group $G'$ with respect to $*$:

\begin{table}[!ht] 
\begin{tabular}{l|llllll}
	$*$ & $e$ & $a$ & $b$ & $ab$ & $aba$ & $abab$	\\ \hline
	$e$ & $e$ & $a$ & $b$ & $ab$ & $aba$ & $abab$    	\\
	$a$ & $a$ & $e$ & $ab$ & $b$ & $abab$ & $aba$ 		\\
	$b$ & $b$ & $abab$ & $e$ & $aba$ & $ab$ & $a$ 		\\
	$ab$ & $ab$ & $aba$ & $a$ & $abab$ & $b$ & $e$ 		\\
	$aba$ & $aba$ & $ab$ & $abab$ & $a$ & $e$ & $b$ 	\\
	$abab$ & $abab$ & $b$ & $aba$ & $e$ & $a$ & $ab$	\\
\end{tabular}
\centering
\caption{Operation table for $G'$ under $*$}
\label{t2}
\end{table}

Now, define the bijection $\phi:G \to G'$ as follows:

\begin{align*}
	r_3 & \mapsto e \\
	r_1 & \mapsto ab \\
	r_2 & \mapsto abab \\
	f_1 & \mapsto b \\
	f_2 & \mapsto aba \\
	f_3 & \mapsto a
\end{align*}

Then, $\phi$ is a one-to-one correspondence
from elements of $G$ that satsify the defining equations of $G'$
to the elements of $G'$ that satisfy themselves satisfy those equations:

\begin{align*}
	\phi(a^2) = \phi(e) = r_3 = f_1 \circ f_1 \\
	\phi(b^2) = \phi(e) = r_3 = f_3 \circ f_3 \\
	\phi(aba) = \phi(bab) = \phi(a) \phi(b) \phi(a) = \phi(b) \phi(a) \phi(b) \\
	 = f_1 \circ f_3 \circ f_1 = f_3 \circ f_1 \circ f_3 = f_2
\end{align*}

Since we have a bijection between $G$ and $G'$
where corresponding elements satisfy the
same defining equations, tables \ref{t1} and \ref{t2}
are actually redudant and we can immediately
conclude that $G \cong G'$

\section{Chapter 9, Exercise I3}

Suppose $G$ is a group and $a \in G$

\begin{thm} \label{thm4}
	$f:G \to G$ such that $f(x) = axa^{-1}$ is an automorphism of $G$.
\end{thm}

\begin{proof}
	Let $x,y \in G$.
	Then, $f(x) = axa^{-1}$ and $f(y) = aya^{-1}$.
	Furthermore, $f(xy) = axya^{-1}$ and $f(x)f(y) = axa^{-1}aya^{-1}$.
	Since $axa^{-1}aya^{-1} = axeya^{-1} = axya^{-1}$,
	we have:
	\begin{align} \label{eq10}
		f(xy) = f(x)f(y)
	\end{align}

	Now suppose that there exists some $z \in G$ where $f(z) = f(x) = axa^{-1}$.
	We also must have, $f(z) = aza^{-1}$ by the definition of $f$,
	so we have $aza^{-1} = axa^{-1}$.
	Multiplying both sides of that on the right by $a$
	gives us $az = ax$,
	and multiplying both sides of that on the left by $a^{-1}$
	gives us $z = x$.
	Therefore, $f$ is injective.

	Let $d$ be some element of $G$ and let $p = a^{-1}da$.
	Notice that $f(p) = aa^{-1}daa^{-1} = ede = d$, 
	therefore $g$ is surjective.

	Since $f$ is injective and surjective,
	it is bijective,
	and since $f$ is a bijection that satisfies
	equation \ref{eq10},
	we have that $f$ is an isomorphism from $G$ to itself,
	or in other words, $f$ is an automorphism of $G$.
	This proves theorem \ref{thm4}.
\end{proof}

\section{Chapter 10, Exercise B2}

Let $10 \in \ints_{25}$

Consider the following equation:
\begin{align} \label{eq12}
	\sum_{i=1}^n 10 = 0
\end{align}

Since the first natural number that satisfies equation \ref{eq12}
is $25$, we have that $\ord(10) = 25$.

\section{Chapter 10, Exercise B3}

Let $6 \in \ints_{16}$

Consider the following equation:
\begin{align} \label{eq13}
	\sum_{i=1}^n 6 = 0
\end{align}

Since the first natural number that satisfies equation \ref{eq13}
is $24$, we have that $\ord(6) = 24$.

\section{Chapter 10, Exercise C4}

Suppose $G$ is a group with operator $*$ and $a,b \in G$

\begin{thm} \label{thm8}
	$\ord(a) = \ord(bab^-1)$
\end{thm}

\begin{proof}
	Let $n = \ord(a)$.
	Then, $a^n = e$.
	Consider the product:
	\begin{align} \label{eq14}
	\prod_{i=1}^n bab^{-1}  = \underbrace{bab^{-1} bab^{-1} ... bab^{-1}}_\text{$n$ times}
	\end{align}
	Since each $b^{-1}b$ in equation \ref{eq14} can be replaced with $e$,
	we can rewrite it like so:
	\begin{align} \label{eq15}
		\prod_{i=1}^n bab^{-1}  = b\underbrace{aa .. a}_\text{$n$ times}b^{-1}
		= ba^nb^{-1}
	\end{align}
	Then, since $a^n = e$, we have:
	\begin{align} \label{eq16}
		\prod_{i=1}^n bab^{-1} = e
	\end{align}

	Now suppose there exists some $m \le n$ where:
	\begin{align} \label{eq17}
		\prod_{i=1}^m bab^{-1} = e
	\end{align}

	Then:
	\begin{align} \label{eq18}
		\prod_{i=1}^m bab^{-1}  = b\underbrace{aa .. a}_\text{$m$ times}b^{-1} & = e \\
		ba^mb^{-1} & = e \\
		ba^m & = e \\
		a^m & = e \iff m = n
	\end{align}

	Thus $n$ is the smallest integer such that $(bab^{-1})^n = e$
	so $\ord(bab^{-1}) = n$ and finally $\ord(a) = \ord(bab^{-1})$.
	This proves theorem \ref{thm8}.

\end{proof}

\section{Chapter 10, Exercise C5}

Suppose $G$ is a group with operator $*$ and $a\in G$

\begin{thm} \label{thm9}
	$\ord(a) = \ord(a^{-1})$
\end{thm}

\begin{proof}
	Let $n = \ord(a)$.
	Then, we have $a^n = \underbrace{aa ..a}_{\text{$n$ times}} = e$.
	Now consider exponentiation of $a^{-1}$.
	We have in general that $(a^{-1})^m = (a^m)^{-1}$ for some integer $m$.
	Therefore, $(a^{-1})^n = (a^n)^{-1} = e^{-1} = e$.
	There is no other smaller positive integer that satisfies $a^n = e$
	by the definition of the order of an element of a group,
	therefore $\ord(a^{-1}) = n$ and $\ord(a) = \ord(a^{-1})$.
	This proves theorem \ref{thm9}.
\end{proof}

\section{Chapter 10, Exercise D5}

Supose $G$ is a group with an element $a$ that has finite order.

Let $n,r,s$ be some integers.

\begin{thm} \label{thm10}
	$\ord(a) = n \land a^r = a^s \implies \exists k \in \ints \ni nk = r-s$
	i.e. $n$ is a factor of $r - s$.
\end{thm}

\begin{proof}
	First we note that $n > 0$ must hold since $\ord(a) = n$
	only holds when $n$ is positive.
	Since $r,s,n \in \ints \land n > 0$,
	we can apply the division algorithm.
	Let $p,q$ be the unique integers such that $r = np + q$ and $0 \le q < n$ and
	let $x,y$ be the unique integers such that $s = nx + y$ and $0 \le y < n$.
	Then:
	\begin{align} \label{eq19}
		r - s = (np + q) - (nx + y) = n(p - x) + (q - y)
	\end{align}
	So we have:
	\begin{align*}
		a^r = & a^s \\
		a^{np + q} = & a^{nx + y} \\
		a^{np}a^q = & a^{nx}a^y \\
		(a^n)^pa^q = & (a^n)^x a^y \\
		e^p a^q = & e^x a^y \\
		a^q = & a^y
	\end{align*}
	Since $0 \le q < n \land 0 \le y < n$,
	we must have that $q = y$ and then we can simpify equation \ref{eq19}:
	\begin{align} \label{eq20}
		r - s = n(p - x)
	\end{align}

	Since $(p - x) \in \ints$, we have some integer $z = p - x$ such that
	$nz = r - q$, or in other words,
	$n$ is a factor of $r - s$.
	This proves theorem \ref{thm10}.
\end{proof}

\section{Isomorphism involving some matricies}

Suppose we have the following matrices:

\begin{align}
	I = \glxmatrix{1}{0}{0}{0}{1}{0}{0}{0}{1} \;
	A = \glxmatrix{1}{0}{0}{0}{0}{1}{0}{1}{0} \;
	B = \glxmatrix{0}{0}{1}{1}{0}{0}{0}{1}{0} \\
	C = \glxmatrix{0}{1}{0}{1}{0}{0}{0}{0}{1} \;
	D = \glxmatrix{0}{1}{0}{0}{0}{1}{1}{0}{0} \;
	E = \glxmatrix{0}{0}{1}{0}{1}{0}{1}{0}{0} 
\end{align}

Define the set $G = \{ I, A, B, C, D, E \}$.
Define the operation $*$ on $G$ as $A * B = BA$

Then, table \ref{t3} is the operation table for $G$.

\begin{table}[!ht] 
\begin{tabular}{l|llllll}
	$*$ & $I$ & $D$ & $E$ & $A$ & $B$ & $C$	\\ \hline
	$I$ & $I$ & $D$ & $E$ & $A$ & $B$ & $C$ \\
	$D$ & $D$ & $E$ & $I$ & $C$ & $A$ & $B$ \\
	$E$ & $E$ & $I$ & $D$ & $B$ & $C$ & $A$ \\
	$A$ & $A$ & $C$ & $B$ & $I$ & $E$ & $D$ \\
	$B$ & $B$ & $A$ & $C$ & $D$ & $I$ & $E$ \\
	$C$ & $C$ & $B$ & $A$ & $E$ & $D$ & $I$	\\
\end{tabular}
\centering
\caption{Operation table for $G$ under $*$}
\label{t3}
\end{table}

Then, utilizing the notation defined by equations \ref{topperm} and \ref{botperm}
in section \ref{permsect} and considering table \ref{t1},
define the bijection $\psi:S_3 \to G$ as follows:

\begin{align*}
	r_3 & \mapsto I \\
	r_1 & \mapsto D \\
	r_2 & \mapsto E \\
	f_1 & \mapsto A \\
	f_2 & \mapsto B \\
	f_3 & \mapsto C
\end{align*}

We can see by inspection of tables \ref{t1} and \ref{t3}
that $\psi$ is an isomorphism from $S_3$ to $G$,
therefore $S_3 \cong G$.

\section{Find the order}

Suppose $1 \in \reals^*$

Then, $1^1 = e$ since $1 = e$, therefore $\ord(1) = 1$ with respect to multiplication.

Now suppose $1 \in \reals$.

Consider the following equation:
\begin{align} \label{eq21}
	\sum_{i=1}^n 1 = 0
\end{align}

There is no value of $n$ that will satisfy this equation,
therefore $\ord(1) = \infty$, i.e. $1$ has infinite order
with respect to addition.



\section{Find the order again}

Suppose $A = \glmatrix{1}{1}{0}{1} \in \gltwo$.

Consider the following equation:
\begin{align} \label{eq22}
	\prod_{i=1}^n A = \glmatrix{1}{n}{0}{1}
\end{align}

Since $n$ never has the value $0$,
there does not exist a positive value of $n$ such that
the right hand side of equation \ref{eq22}
is $\glmatrix{1}{0}{0}{1}$.

Therefore, $\ord(A) = \infty$, i.e. $A$ has infinite order.

\section{Find the order yet again}

Suppose $f(x) = \frac{1}{1-x} \in S_A$ where $A = \reals - \{ 0, 1 \}$

We want to find $\ord(f)$, i.e the smallest positive integer such that $f^n(x) = e$,
--- where $e$ is the identity function ---
or infinity if such a numer does not exist.

First, we find $f^2(x)$:
\begin{align}
	\frac{1}{1-\frac{1}{1-x}} \\
	\frac{1}{\frac{1 - x - 1}{1 - x}} \\
	\frac{1}{\frac{ -x }{1 - x}} \\
	\frac{1 - x}{-x} \\
	\frac{x - 1}{x} \\
	1 - \frac{1}{x} 
\end{align}

Then, we plug out simplified $f^2(x)$ back into $f$ and simplify to find $f^3(x)$:

\begin{align}
	\frac{1}{1-(1 - \frac{1}{x})} \\
	\frac{1}{\frac{1}{x}} \\
	x
\end{align}

Therefore $f^3(x) = x$ and then by inspection, we have $\ord(f) = 3$.

\section{Subgroup problems}

This problem contains three subproblems.
I have divided these among the following three subsections.

\subsection{Permutations} \label{sub1}

Suppose $G_1 = \{ f \in S_{\reals}\ |\ f(x) = ax + b \ni a \neq 0 \land a,b \in \reals \}$

\begin{thm} \label{thm5}
	$G_1$ is a subgroup of $S_{\reals}$ with respect to function composition.
\end{thm} 

\begin{proof}
	Let $f,g \in G_1$ such that
	$f(x) = ax + b$ for some nonzero real $a$ and some real $b$ and
	$g(x) = cx + d$ for some nonzero real $c$ and some real $d$
	Then, we have
	$(g \circ f)(x) = g(f(x)) = c(ax + b) + d = cax + (cb + d)$.
	Since $c \neq 0 \neq a \implies ca \neq 0$
	and $ca \in \reals$ as well as $(cb + d) \in \reals$,
	we have that $(g \circ f)(x) \in G_1$,
	therefore $G_1$ is closed under $\circ$.

	Let $h(x) = a^{-1}x + (-ba^{-1})$.
	$a^{-1}$ is defined since $a \neq 0$
	and $a^{-1} \in \reals \land -ba^{-1} \in \reals \implies h(x) \in G$.
	Then, we see that:
	\begin{align*}
		(h \circ f)(x) & = a^{-1}(ax + b) + -ba^{-1} \\
			       & = x + ba^{-1} -ba^{-1} \\
			       & = x \\
		(f \circ g)(x) & = a(a^{-1}x + -ba^{-1}) + b \\
			       & = x + -b + b \\
			       & = x
	\end{align*}
	Since $(f \circ h)(x) = (h \circ f)(x) = x \in G_1$ is the identity function,
	we have that the identity of $S_{\reals}$ is in $G_1$
	and also that $h(x) = f^{-1}(x)$,
	so it follows that every element in $G_1$ has its inverse in $G_1$.

	Finally, since $G_1$ is a closed with respect to $\circ$ and
	since every element in $G_1$ has its inverse in $G_1$,
	we have that $G_1$ is a subgroup of $S_{\reals}$.
	This proves theorem \ref{thm5}.
\end{proof}

\subsection{The General Linear Group} \label{sub2}

Suppose $G_2 = \Big\{\glmatrix{a}{b}{0}{1} \in \gltwo \ |\ a \neq 0 \Big\}$

\begin{thm} \label{thm6}
	$G_2$ is a subgroup of $\gltwo$ with respect to matrix multiplication.
\end{thm}

\begin{proof}
	Let $A = \glmatrix{a}{b}{0}{1}$ and $B = \glmatrix{c}{d}{0}{1}$
	be two elements of $G_2$.

	Then, we see that $AB = \glmatrix{ac}{ad + b}{0}{1}$.
	Since $a \neq 0 \neq c \implies ac \neq 0$ and
	since $ad \in \reals \land ad + b \in reals$,
	we must have that $AB \in G_2$ by its definition.
	Therefore, $G_2$ is closed under matrix multiplication.

	Let $C = \glinverse{a}{b}{0}{1} = \glmatrix{a^{-1}}{-ba^{-1}}{0}{1}$,
	then, we can see by inspection that $C \in G_2$.
	We then have $CA = AC = \glmatrix{1}{0}{0}{1} = I \in G_2$,
	so we also have that $C = A^{-1}$
	and that $e = I$, the identity of $\gltwo$, is in $G_2$.

	Finally, since $G_2$ is a closed with respect to
	matrix multiplication and
	since every element in $G_2$ has its inverse in $G_2$,
	we have that $G_2$ is a subgroup of $\gltwo$.
	This proves theorem \ref{thm6}.
	
\end{proof}

\subsection{Isomorphism}

Define the function $T: G_1 \to G_2$ as $T(f) = \glmatrix{a}{b}{0}{1}$
where $G_1$ and $G_2$ denote the same groups as in subsections \ref{sub1} and \ref{sub2}
and $f \in G_1$ is some $f(x) = ax + b$ for some nonzero real $a$ and some real $b$.

\begin{thm} \label{thm7}
	$G_1 \cong G_2$
\end{thm}

\begin{proof}
	Let $f,g \in G_1$ be such that $f(x) = ax + b$ and $g(x) = cx + d$
	where $a,b,c,d \in \reals \land a \neq 0 \neq c$.

	\sloppy
	Then, $T(f) = \glmatrix{a}{b}{0}{1} \land T(g) = \glmatrix{c}{d}{0}{1}$.
	We see that $T(f)T(g) = \glmatrix{ac}{ad + b}{0}{1}$,
	and also that $T(f \circ g) = T((ac)x + (ad + b)) = \glmatrix{ac}{ad + b}{0}{1}$,
	therefore:
	\begin{align} \label{eq11}
		T(f \circ g) = T(f)T(g)
	\end{align}

	Let $h \in G_1$ be defined as $h(x) = px + q$ for some $a,b \in \reals \ni a \neq 0$
	such that $T(h) = T(f)$. Then, we must have:
	\begin{align}
		\glmatrix{p}{q}{0}{1} = \glmatrix{a}{b}{0}{1} \iff p = a \land q = b
	\end{align}
	Therefore $h = f$ and $T$ is injective.
	
	Let $Q = \glmatrix{y}{z}{0}{1}$ for some nonzero real $y$ and some real $z$.
	We can see by inspection that $Q \in G_2$, and
	we can construct a $\phi \in G_1$ where $\phi(x) = yx + z$ and it is
	readily apparent that $T(\phi) = Q$.
	Therefore $T$ is surjective.

	Since $T$ is injective and surjective, it is bijective.
	Then, since $T$ is a bijection that satisfies equation \ref{eq11},
	$T$ is an isomorphism from $G_1$ to $G_2$.
	Finally, since there exists some isomorphism from $G_1$ to $G_1$,
	we have that $G_1$ is isomorphic to $G_2$, or $G_1 \cong G_2$.
	This proves theorem \ref{thm7}.


\end{proof}

\end{document}
