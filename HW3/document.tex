\documentclass[12pt]{article}
\usepackage{url,amsmath,amsthm,enumitem,amsfonts,tikz,verbatim,amssymb, wasysym,multicol, nccmath}
\usepackage[makeroom]{cancel}
\usetikzlibrary{arrows}

\title{Abstract Algebra: Homework \#3}
\date{Wednesday 10 June 2020}
\author{Joel Savitz}

\newcommand{\reals}{\mathbb{R}}
\newcommand{\ainverse}{\frac{1}{a_1 a_4 - a_2 a_3}\begin{bmatrix} a_4 & -a_2 \\ -a_3 & a_1 \\ \end{bmatrix}}
\newcommand{\ints}{\mathbb{Z}}
\newcommand{\contreals}{\mathcal{C}(\reals)}
\newtheorem{thm}{Theorem}
\newtheorem{cnt}{Counterexample}

\begin{document}

\maketitle

Note: for the scope of this document, let $\ni$ denote ``such that''.

\section{Chapter 7, Exercise A1}

Suppose we have some $f,g,h \in S_6$ such that:
\begin{align}
	f = \binom{1\ 2\ 3\ 4\ 5\ 6}{6\ 1\ 3\ 5\ 4\ 2} \\
	g = \binom{1\ 2\ 3\ 4\ 5\ 6}{2\ 3\ 1\ 6\ 5\ 4} \\
	h = \binom{1\ 2\ 3\ 4\ 5\ 6}{3\ 1\ 6\ 4\ 5\ 2}
\end{align}

Then, we have the following inverse functions and compositions of functions:

\begin{align}
	f^{-1} = & \binom{1\ 2\ 3\ 4\ 5\ 6}{2\ 6\ 3\ 5\ 4\ 1} \\
	g^{-1} = & \binom{1\ 2\ 3\ 4\ 5\ 6}{3\ 1\ 2\ 6\ 5\ 4} \\
	h^{-1} = & \binom{1\ 2\ 3\ 4\ 5\ 6}{2\ 6\ 1\ 4\ 5\ 3} \\
	f \circ g = & \binom{1\ 2\ 3\ 4\ 5\ 6}{1\ 3\ 6\ 2\ 4\ 5} \\
	g \circ f = & \binom{1\ 2\ 3\ 4\ 5\ 6}{4\ 2\ 1\ 5\ 6\ 3}
\end{align}

\section{Chapter 4, Exercise A4}

\pagebreak
\end{document}
