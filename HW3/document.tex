\documentclass[12pt]{article}
\usepackage{url,amsmath,amsthm,enumitem,amsfonts,tikz,verbatim,amssymb, wasysym,multicol, nccmath}
\usepackage[makeroom]{cancel}
\usetikzlibrary{arrows}

\title{Abstract Algebra: Homework \#3}
\date{Wednesday 10 June 2020}
\author{Joel Savitz}

\newcommand{\reals}{\mathbb{R}}
\newcommand{\ainverse}{\frac{1}{a_1 a_4 - a_2 a_3}\begin{bmatrix} a_4 & -a_2 \\ -a_3 & a_1 \\ \end{bmatrix}}
\newcommand{\ints}{\mathbb{Z}}
\newcommand{\contreals}{\mathcal{C}(\reals)}
\newtheorem{thm}{Theorem}
\newtheorem{cnt}{Counterexample}

\begin{document}

\maketitle

Note: for the scope of this document, let $\ni$ denote ``such that''.

\section{Chapter 7, Exercise A1}

Suppose we have some $f,g,h \in S_6$ such that:
\begin{align}
	f = \binom{1\ 2\ 3\ 4\ 5\ 6}{6\ 1\ 3\ 5\ 4\ 2} \\
	g = \binom{1\ 2\ 3\ 4\ 5\ 6}{2\ 3\ 1\ 6\ 5\ 4} \\
	h = \binom{1\ 2\ 3\ 4\ 5\ 6}{3\ 1\ 6\ 4\ 5\ 2}
\end{align}

Then, we have the following inverse functions and compositions of functions:

\begin{align}
	f^{-1} = & \binom{1\ 2\ 3\ 4\ 5\ 6}{2\ 6\ 3\ 5\ 4\ 1} \\
	g^{-1} = & \binom{1\ 2\ 3\ 4\ 5\ 6}{3\ 1\ 2\ 6\ 5\ 4} \\
	h^{-1} = & \binom{1\ 2\ 3\ 4\ 5\ 6}{2\ 6\ 1\ 4\ 5\ 3} \\
	f \circ g = & \binom{1\ 2\ 3\ 4\ 5\ 6}{1\ 3\ 6\ 2\ 4\ 5} \\
	g \circ f = & \binom{1\ 2\ 3\ 4\ 5\ 6}{4\ 2\ 1\ 5\ 6\ 3}
\end{align}

\section{Chapter 7, Exercise B2}

Suppose we have some $f \in S_6$ such that:
\begin{align}
	f = & \binom{1\ 2\ 3\ 4\ 5\ 6}{2\ 3\ 4\ 1\ 6\ 5}
\end{align}

Then, we generate a cyclic subgroup of $S_6$ containing the following elements. 
\begin{align}
	f^2 = ff = & \binom{1\ 2\ 3\ 4\ 5\ 6}{3\ 4\ 1\ 2\ 5\ 6} \\
	f^3 = fff = & \binom{1\ 2\ 3\ 4\ 5\ 6}{4\ 1\ 2\ 3\ 6\ 5} \\
	f^4 = ffff = & \binom{1\ 2\ 3\ 4\ 5\ 6}{1\ 2\ 3\ 4\ 5\ 6} = e \\
	f^5 = ff^4  = fe = & f
\end{align}

Let $A = \{ f, f^2, f^3, f^4 \}$.

It is readily checked that $A$
is a $4^{th}$ order cyclic subgroup of $S_6$
under the operation of
function composition.

\section{Chapter 7, Exercise B3}

Define a subset of $S_5$ as $A = \{ e, f, g, h \}$
with elements defines as follows:

\begin{align}
	e = & \binom{1\ 2\ 3\ 4\ 5}{1\ 2\ 3\ 4\ 5} \\
	f = & \binom{1\ 2\ 3\ 4\ 5}{2\ 1\ 3\ 4\ 5} \\
	g = & \binom{1\ 2\ 3\ 4\ 5}{1\ 2\ 3\ 5\ 4} \\
	h = & \binom{1\ 2\ 3\ 4\ 5}{2\ 1\ 3\ 5\ 4}
\end{align}

Table \ref{t1} describes the behavior of $A$
under function composition.

We see by inspection of table \ref{t1} that
$A$ is closed under $\circ$,
contains the identity of $S_6$,
contains an inverse for each element,
and contains absolutely zero non-commutative elements.

Then, we conclude that $A$ is a subgroup of $S_6$,
and furthermore we conclude that $A$ is an
Abelian subgroup of $S_6$
since any element of $A$ commutes with
any other element of $A$.


\begin{table}[!ht] 
\begin{tabular}{l|llll}
	$\circ$ & $e$ & $f$ & $g$ & $h$ \\ \hline
	$e$ & $e$ & $f$ & $g$ & $h$ \\
	$f$ & $f$ & $e$ & $h$ & $g$ \\
	$g$ & $g$ & $h$ & $e$ & $f$ \\
	$h$ & $h$ & $g$ & $f$ & $e$ \\
\end{tabular}
\centering
\caption{Operation table for $A$ under $\circ$}
\label{t1}
\end{table}

\section{Chapter 7, Exercise C2}

Suppose $A = \{ x \in \reals \ni x \neq 0 \}$
and $G = \{e, f, g, h \}$ where:

\begin{align}
	e(x) =  x \land
	f(x) = \frac{1}{x} \land
	g(x) = -x \land
	h(x) = \frac{-1}{x}
\end{align}

Table \ref{t2} describes the behavior of $A$
under function composition.

\begin{table}[!ht] 
\begin{tabular}{l|llll}
	$\circ$ & $e$ & $f$ & $g$ & $h$ \\ \hline
	$e$ & $e$ & $f$ & $g$ & $h$ \\
	$f$ & $f$ & $e$ & $h$ & $g$ \\
	$g$ & $g$ & $h$ & $e$ & $f$ \\
	$h$ & $h$ & $g$ & $f$ & $e$ \\
\end{tabular}
\centering
\caption{Operation table for $A$ under $\circ$}
\label{t2}
\end{table}

By inspection of table \ref{t2},
we see that $G$ is closed under $\circ$,
that the identity of $S_A$ is in $G$,
and that every element in $G$
has its inverse under $\circ$ in $G$,
so we conclude that $G$ is a subgroup of $S_A$.

\textbf{Remark 1:}
Tables \ref{t1} and \ref{t2} are identical,
though the elements of $G$
are defined in terms
of an enirely different $A$.

In fact,
if we continue to let $G$ denote
the group in this problem
and if we let $G'$ denote
the $G$ from the previous problem,
and let $e', f', g', h'$ denote the $e,f,g,h$ from the previous problem
and if we define a $\phi:G \to G'$
such that $\phi(e) = e', \phi(f) = f', \phi(g) = g', \phi(h) = h'$,
we observe the linearity of $\phi$,
that $(\forall x,y \in G)(\phi(x \circ y) = \phi(x) \circ \phi(y))$
and that $\phi$ is a bijection,
so we have group isomorphism between $G$ and $G'$,
or $G \cong G'$.

\section{Chapter 7, Exercise F2}

Suppose $H$ is the group of symmetries of the rectangle.

Denote the full $2\pi$ radian rotation
of the rectangle
as $r_e$,
since it is the identity of $G$

Denote the $\pi$ radian rotation
of the rectangle
as $r_\pi$.

Denote the horizontal and vertical flips
of the rectangle
as $f_h$ and $f_v$ respectively.

More precisely, we define the following four permutations:
\begin{align}
	r_e = \binom{1\ 2\ 3\ 4}{1\ 2\ 3\ 4} \land
	r_\pi = \binom{1\ 2\ 3\ 4}{3\ 4\ 1\ 2} \land
	f_h  = \binom{1\ 2\ 3\ 4}{4\ 3\ 2\ 1} \land
	f_v  = \binom{1\ 2\ 3\ 4}{2\ 1\ 4\ 3}
\end{align}

Then, table \ref{t3} describes the behavior of $H$
under function compositon

\begin{table}[!ht] 
\begin{tabular}{l|llll}
	$\circ$ & $r_e$ & $r_\pi$ & $f_h$ & $f_v$ \\ \hline
	$r_e$ & $r_e$ & $r_\pi$ & $f_h$ & $f_v$ \\
	$r_\pi$ & $r_\pi$ & $r_e$ & $f_v$ & $f_h$ \\
	$f_h$ & $f_h$ & $f_v$ & $r_e$ & $r_\pi$ \\
	$f_v$ & $f_v$ & $f_h$ & $r_\pi$ & $r_e$ \\
\end{tabular}
\centering
\caption{Operation table for $H$ under $\circ$}
\label{t3}
\end{table}


\textbf{Remark 2:} Define a function $\psi:H \to G$,
where $G$ is defined as in the previous exercise,
such that $\psi(r_e) = e$,
$\psi(r_\pi) = f$,
$\psi(f_h) = g$, and
$\psi(f_v) = h$.
We see by inspection that $\psi$ is bijective
and it is readily checked
via tables \ref{t2} and \ref{t3}
that $(\forall x,y \in H)(\psi(x) \circ \psi(y) = \psi(x \circ y))$ holds.
Then, we have $H \cong G$,
and by transitivity of isomorphism
we have that tables \ref{t1}, \ref{t2}, and \ref{t3}
describe isomorphic groups.

\section{Chapter 7, Exercise H3}

Suppose some set $B$ is a subset of a finite set $A$.

Let $G$ be the subset of $S_A$ where $(\forall x \in B)(f(x) \in B)$.

\begin{thm} \label{thm1}
	$G$ is a subgroup of $A$.
\end{thm}

\begin{proof}
	We have by the definition of $G$ that for any $f \in G$,
	we have $f(x) \in B$ for any $x \in B$, so for any two
	$f,g \in B$, we must also have $(f \circ g)(x) = f(g(x)) \in B$,
	so $G$ is closed under $\circ$.

	Then, for some $f \in G$, we have $f(x) \in B$ for any $x \in B$,
	and we have that any permutation is a one-to-one correspondence,
	so if we simply reverse the direction of the mappings for $f$,
	we have a $g$ where $g(f(x)) = x$ and since $x \in B$ and $f(x) \in B$
	we must also have $g \in B$ and we see that $g = f^{-1}$.
	So, every element of $G$ has its inverse in $G$ and then
	it follows immediately that the identity of $S_A$ is in $G$.
	
	Since $G$ is closed under function composition and every element
	has an inverse, we conclude that $G$ is a subgroup of $S_A$.
	This proves theorem \ref{thm1}.
\end{proof}

\section{Chapter 8, Exercise A1(f)}

Suppose $f = (6\ 1\ 4\ 8)(2\ 3\ 4\ 5)(1\ 2\ 4\ 9\ 3) \in S_9$.

\medskip
\noindent We can rewrite $f$ as $\binom{1\ 2\ 3\ 4\ 5\ 6\ 7\ 8\ 9}{3\ 5\ 4\ 9\ 2\ 1\ 7\ 6\ 8}$.

\section{Chapter 8, Exercise A2(d)}

Suppose $f = \binom{1\ 2\ 3\ 4\ 5\ 6\ 7\ 8\ 9}{9\ 8\ 7\ 4\ 3\ 6\ 5\ 1\ 2} \in S_9$.

\medskip
\noindent We can rewrite $f$ as $(1\ 9\ 2\ 8)(3\ 7\ 5)$.

\section{Chapter 8, Exercise A3(b)}

Suppose $f = (4\ 1\ 6)(8\ 2\ 3\ 5)$

\medskip
\noindent We can rewrite $f$ as $(6\ 1)(6\ 4)(5\ 3)(5\ 2)(5\ 8)$.


\section{Chapter 8, Exercise B1(b)}

Let $\alpha = (1\ 2\ 3\ 4)$.

Then:
\begin{align}
	\alpha^{-1} = & (1\ 4\ 3\ 2) \\
	\alpha^2 = & (1\ 3)(2\ 4) \\
	\alpha^3 = & (1\ 4\ 3\ 2) \\
	\alpha^4 = & (1)(2)(3)(4) \\
	\alpha^5 = & (1\ 2\ 3\ 4) = \alpha
\end{align}

\section{Chapter 8, Exercise C1(c)}

Let $f = (1\ 2)(7\ 6)(3\ 4\ 5)$.

If we rewrite $f$ as $(1\ 2)(7\ 6)(5\ 4)(5\ 3)$
we can see by inspection
that $f$ is an even permutation
since it is the product of an even number
of transpositions.

\section{Symmetries of the equilateral triangle}

Let $G \subseteq S_3$ be the symmetries of the equilateral triangle.

Denote the rotations by $r_1, r_2, r_3$ for a rotation $r_n$ of $\frac{2 n\pi}{3}$ radians.
Denote the flips about each axis of symmetry by $f_1, f_2, f_3$.

More precisely, we define the following permutations:
\begin{align}
	r_1 = & \binom{1\ 2\ 3}{2\ 3\ 1} &
	r_2 = & \binom{1\ 2\ 3}{3\ 1\ 2} &
	r_3 = & \binom{1\ 2\ 3}{1\ 2\ 3} \\
	f_1 = & \binom{1\ 2\ 3}{1\ 3\ 2} &
	f_2 = & \binom{1\ 2\ 3}{3\ 2\ 1} &
	f_3 = & \binom{1\ 2\ 3}{2\ 1\ 3}
\end{align}


Table 4 describes the behavior of $G$ under function composition.

\begin{table}[!ht] 
\begin{tabular}{l|llllll}
	$\circ$ & $r_3$ & $r_1$ & $r_2$ & $f_1$ & $f_2$ & $f_3$	\\ \hline
	$r_3$ & $r_3$ & $r_1$ & $r_2$ & $f_1$ & $f_2$ & $f_3$   \\
	$r_1$ & $r_1$ & $r_2$ & $r_3$ & $f_3$ & $f_1$ & $f_2$ 	\\
	$r_2$ & $r_2$ & $r_3$ & $r_1$ & $f_2$ & $f_3$ & $f_1$ 	\\
	$f_1$ & $f_1$ & $f_3$ & $f_2$ & $r_3$ & $r_2$ & $r_1$ 	\\
	$f_2$ & $f_2$ & $f_1$ & $f_3$ & $r_2$ & $r_3$ & $r_2$ 	\\
	$f_3$ & $f_3$ & $f_2$ & $f_1$ & $r_1$ & $r_2$ & $r_3$	\\
\end{tabular}
\centering
\caption{Operation table for $G$ under $\circ$}
\label{t4}
\end{table}

\textbf{Remark 3:} We see $G = S_3$ by the observation that $|G| = 3! = 6 = |S_3|$.

\section{A subgroup of $S_4$}

Suppose $G = \{ \binom{1\ 2\ 3\ 4}{x\ y\ z\ 4} \in S_4 \ni x,y,z \in \{1,2,3\} \}$.

Then, we ``arbitrarily" denote the elements of $G$ as follows:

\begin{align}
	r_1 = & \binom{1\ 2\ 3\ 4}{2\ 3\ 1\ 4} &
	r_2 = & \binom{1\ 2\ 3\ 4}{3\ 1\ 2\ 4} &
	r_3 = & \binom{1\ 2\ 3\ 4}{1\ 2\ 3\ 4} \\
	f_1 = & \binom{1\ 2\ 3\ 4}{1\ 3\ 2\ 4} &
	f_2 = & \binom{1\ 2\ 3\ 4}{3\ 2\ 1\ 4} &
	f_3 = & \binom{1\ 2\ 3\ 4}{2\ 1\ 3\ 4}
\end{align}

By sheer coincidence, table \ref{t4} describes the behavior of $G$
under function composition.

\section{Chapter 8, Exercises C2 and C3}

\begin{thm} \label{thm2}
	If two permutations are both even or both odd,
	then their product is even.
\end{thm}

\begin{proof}
	Suppose we have some $\pi \in S_n$
	that can be factored
	into $m$ transpositions.
	We note that $(-1)^m = 1$ iff $m$ is even
	and $(-1)^m = -1$ iff $m$ is odd.
	
	We know that for any two permutations $a,b \in S_n$
	composed of $q$ and $r$ transpositions each respectively,
	if we let $s$ denote the number of transpositions
	of $ab$, we have $(-1)^s = (-1)^q \cdot (-1)^r$.

	Then, suppose we have two even permutations $x$ and $y$
	composed of $i$ and $j$ transpositions respectively.
	We have $(-1)^i \cdot (-1)^j = 1 \cdot 1 = 1$
	which implies that $xy$ must be an even permutation.

	Alternatively, suppose we have two odd permutations $x$ and $y$
	composed of $i$ and $j$ transpositions respectively.
	We have $(-1)^i \cdot (-1)^j = -1 \cdot -1 = 1$
	which implies that $xy$ must be an even permutation.

	Finally, we conclude that if any two permutations are
	either both even or both odd, we must have that
	their product is even.

	This proves theorem \ref{thm2}.
\end{proof}

\section{A subset of $S_4$ consisting of transpositions}

Suppose $G = \{ e, (1\ 2)(3\ 4), (1\ 3)(2\ 4), (1\ 4)(2\ 3) \} \subseteq S_4$.

Denote the elements of $G$ as follows:
\begin{align}
	e = & e \\
	f = & (1\ 2)(3\ 4) \\
	g = & (1\ 3)(2\ 4) \\
	h = & (1\ 4)(2\ 3)
\end{align}

\begin{table}[!ht] 
\begin{tabular}{l|llll}
	$\circ$ & $e$ & $f$ & $g$ & $h$ \\ \hline
	$e$ & $e$ & $f$ & $g$ & $h$ \\
	$f$ & $f$ & $e$ & $h$ & $g$ \\
	$g$ & $g$ & $h$ & $e$ & $f$ \\
	$h$ & $h$ & $g$ & $f$ & $e$ \\
\end{tabular}
\centering
\caption{Operation table for $G$ under $\circ$}
\label{t5}
\end{table}

Hmm, this looks suspicously similar to table \ref{t2}.

Anyway, we confidently conclude that
due to the fact that $G$
is closed under $\circ$
and has an inverse under $\circ$
for every element in the set,
including the identity of $S_4$,
$G$ is a subgroup of $S_4$.

\end{document}
