\documentclass[12pt]{article}
\usepackage{url,amsmath,amsthm,enumitem,amsfonts,tikz,verbatim,amssymb, wasysym,multicol}
\usepackage[makeroom]{cancel}
\usetikzlibrary{arrows}

\title{Abstract Algebra: Homework \#1}
\date{Wednesday 27 May 2020}
\author{Joel Savitz}

\newcommand{\reals}{\mathbb{R}}
\newcommand{\ainverse}{\frac{1}{a_1 a_4 - a_2 a_3}\begin{bmatrix} a_4 & -a_2 \\ -a_3 & a_1 \\ \end{bmatrix}}
\newcommand{\ints}{\mathbb{Z}}
\newtheorem{thm}{Theorem}

\begin{document}

\maketitle

Note: for the scope of this document, let $\ni$ denote ``such that''.

\section{Chapter 3: Excercise A1}

Suppose $*$ is defined on $\reals$ as $a * b = a + b + k$ for any $a,b \in \reals$ for some $k \in \reals$.

\begin{thm}
	$\langle \reals, * \rangle$ is a group.
\end{thm}

\begin{proof}
	Let $a,b,$ and $c$ be some arbitrary real numbers.
	Because $a * b = a + b + k \in \reals$, we have that the real numbers are closed under $*$.
	Then, observe that $a * (b * c)  = a * (b + c + k) = a + (b + c + k) + k = (a + b + k) + c + k = (a + b + k) * c = (a * b) * c$, so $*$ is associative.
	We also have $a * -k = a + (-k) + k = -k + a + k = -k * a = a$, so $-k$ is the identity for the real numbers under $*$
	Finally, consider the quantity $-(2k+a)$. Since we have $a * -(2k+a) = a + (-2k) +  (-a) + k = -(2k+a) * a = -k$, that quantity is the inverse of any $a$.
	Since $*$ is closed under the real numbers and $*$ is associative and $-k$ is the identity of $*$ under the real numbers and any real number $a$ has an inverse under $*$ of $-(2k-a)$, $\langle \reals, * \rangle$ is a group.
\end{proof}

\section{Chapter 3: Exercise A3}

Suppose $*$ is defined on $\reals$ as $a * b = a + b + ab$ for any $a,b \in \reals$.

\begin{thm}
	$\langle \reals, * \rangle$ is a group.
\end{thm}

\begin{proof}
	Let $a,b,$ and $c$ be some arbitrary real numbers.
	Because $a * b = a + b + ab \in \reals$, we have that the real numbers are closed under $*$.
	Then, observe that $a * (b * c)  = a * (b + c + bc) = a + b + c + ab + ac + bc + abc = (a + b + ab) * c = (a * b) * c$, so $*$ is associative.
	We also have $a * 0 = a + (0) + 0a = 0 * a = a$, so $0$ is the identity for the real numbers under $*$
	Finally, consider the quantity $\frac{-a}{1+a}$.
	Since we have $a * \frac{-a}{1+a} = a + \frac{-a}{1+a} +  \frac{-a^2}{1+a} = \frac{-a}{1 + a} * a = \frac{a^2+a}{1+a} + \frac{-a}{1+a} + \frac{-a^2}{1+a} = 0$, that quantity is the inverse of any $a$.
	Since $*$ is closed under the real numbers and $*$ is associative and $0$ is the identity of $*$ under the real numbers and any real number $a$ has an inverse under $*$ of $\frac{-a}{1+a}$, $\langle \reals, * \rangle$ is a group.
\end{proof}

\section{Chapter 3: Exercise B1}

Suppose $*$ is defined on $\reals \times \reals$ as $(a,b) * (c,d) = (ad+bc,bd)$ for any $(a,b),(c,d) \in \reals \times \reals$.

\begin{thm}
	$\langle \reals, * \rangle$ is a group.
\end{thm}

\begin{proof}
	Let $(a,b),(c,d),(e,f) \in \reals \times \reals$.
	Consider that $(a,b) * (c,d) = (ad+bc,bd) \in \reals \times \reals$ since $ad+bc \in \reals \land bc \in \reals$. Then, $\reals \times \reals$ is closed under $*$
	Now observe the following equivalence: $\Big((a,b) * (c,d)\Big) * (e,f) = (ad+bc,bd) * (e,f) = (adf+bcf+bde,bdf) = (a,b) * (cf+de,df) = (a,b) * \Big((c,d) * (e,f)\Big)$
	Since $(a,b) * (0,1) = (1a+0b,1b) = (a,b) = (0b + 1a,1b) = (0,1) * (a,b)$, we have that $(0,1)$ is the identity of the real numbers under $*$,
	Because we have $(a,b)*(\frac{-a}{b^2},\frac{1}{b})=(\frac{a}{b}+\frac{-a}{b}, b \cdot \frac{1}{b} = (0,1) = (\frac{-a}{b}+\frac{a}{b},\frac{1}{b}) = (\frac{-a}{b^2},\frac{1}{b}) * (a,b)$, that pair $(\frac{-a}{b^2},\frac{1}{b})$ is the inverse of any real $a$ under $8$.
	Then, since $*$ is a closed associative operation on the real numbers with an identity and an inverse for any element of the real numbers, we have that $\langle \reals, * \rangle$ is a group.
	
\end{proof}

\section{Chapter 3: Exercise D}

Suppose $*$ is defined as an operation on the set $A = \{ I, V, H, D\}$ as follows in table \ref{t1}:

\begin{table}[!ht] 
\begin{tabular}{l|llll}
$*$ & $I$ & $V$ & $H$ & $D$ \\ \hline
$I$ & $I$ & $V$ & $H$ & $D$ \\
$V$ & $V$ & $I$ & $D$ & $H$ \\
$H$ & $H$ & $D$ & $I$ & $V$ \\
$D$ & $D$ & $H$ & $V$ & $I$ \\
\end{tabular}
\centering
\caption{Operation table for $*$ on $A$}
\label{t1}
\end{table}


Then, we have by table \ref{t1} that $*$ is a closed operation on $A$. Given that $*$ is associative, that $I$ is an identity for $*$ by table \ref{t1}, and that any $a \in A$ has the inverse $a \in a$ by table \ref{t1}, we conclude that $\langle A, * \rangle$ is a group.

Furthermore, since we have that $*$ is commutative by table \ref{t1}, $\langle A, * \rangle$ is an Abelian group.


\section{Chapter 3: Exercise E}

Suppose some set $A$ is defined as follows in equation \ref{eq1}:
\begin{equation}
\label{eq1}
	A = \{ I, M_1, M_2, M_3, M_4, M_5, M_6, M_7 \}
\end{equation}

The, suppose the binary operation $*$ is defined on $A$ as follows in table \ref{t2}:

\begin{table}[!ht] 
\begin{tabular}{l|llllllll}
	$*$ & $I$ & $M_1$ & $M_2$ & $M_3$ & $M_4$ & $M_5$ & $M_6$ & $M_7$  \\ \hline
	$I$ & $I$ & $M_1$ & $M_2$ & $M_3$ & $M_4$ & $M_5$ & $M_6$ & $M_7$  \\
	$M_1$ & $M_1$ & $I$ & $M_3$ & $M_2$ & $M_5$ & $M_4$ & $M_7$ & $M_6$  \\
	$M_2$ & $M_2$ & $M_3$ & $I$ & $M_1$ & $M_6$ & $M_7$ & $M_4$ & $M_5$  \\
	$M_3$ & $M_3$ & $M_2$ & $M_1$ & $I$ & $M_7$ & $M_6$ & $M_5$ & $M_4$  \\
	$M_4$ & $M_4$ & $M_6$ & $M_5$ & $M_7$ & $I$ & $M_1$ & $M_2$ & $M_3$  \\
	$M_5$ & $M_5$ & $M_4$ & $M_7$ & $M_6$ & $M_1$ & $M_3$ & $I$ & $M_2$  \\
	$M_6$ & $M_6$ & $M_7$ & $M_4$ & $M_5$ & $M_2$ & $I$ & $M_3$ & $M_1$  \\
	$M_7$ & $M_7$ & $M_6$ & $M_5$ & $M_4$ & $M_3$ & $M_1$ & $M_2$ & $I$  \\
\end{tabular}
\centering
\caption{Operation table for $*$ on $A$}
\label{t2}
\end{table}

\textbf{TODO: fix this table}

\section{A counterexample}

Let $*$ be an operation defined on the set $G = \{ x \in \ints \ni x \neq -1 \}$ defines as $x * y = x + y + xy$.

\begin{thm} \label{thm:6}
	$\langle G, * \rangle$ is not a group.
\end{thm}

\begin{proof}
	Assume for the sake of contradictioion that $G$ has an identity element and call it $e$.
	Then, we must have for any $a \in G$ that $a * e = a = a + e + ae \implies e = -ae \implies a = -1$.
	However, $-1 \not\in G$ and so we reach a contradiction,
	therefore our assumption must be wrong and there does not in fact exist an identity in $G$ under $*$.
	Then, since a group must have an identity element and $G$ has no identity element, $G$ is not a group.
\end{proof}

\section{2 $\times$ 2 invertible matrices}

Let $G = \{A = \begin{bmatrix} a & b \\ c & d \\ \end{bmatrix} \in \reals^{2 \times 2} \ni \exists A^{-1} \in \reals^{2 \times 2} \ni AA^{-1} = A^{-1}A = \begin{bmatrix} 1 & 0 \\ 0 & 1 \\ \end{bmatrix} \}$

\noindent Let $*$ be defined as standard matrix multiplication.

\begin{thm} \label{thm:7}
	$\langle G,* \rangle$ is a group.
\end{thm}

\begin{proof}

Define the following three matrices $A,B,C \in \reals^{2 \times 2}$ as follows:
\begin{align} \label{eq:matrix_defs}
	A = \begin{bmatrix} a_1 & a_2 \\ a_3 & a_4 \\ \end{bmatrix}
	B = \begin{bmatrix} b_1 & b_2 \\ b_3 & b_4 \\ \end{bmatrix}
	C = \begin{bmatrix} c_1 & c_2 \\ c_3 & c_4 \\ \end{bmatrix}
\end{align}

Furthermore, suppose that matrices $A$ and $B$ are nonsingular, i.e. $A,B \in G$

The following equation defines a relationship between these matrices $C = A * B$ and the subsequent equation makes an observation about related determinants.

\begin{align} \label{eq:matrix}
	C = & AB =
	\begin{bmatrix} a_1 & a_2 \\ a_3 & a_4 \\ \end{bmatrix}
	\cdot
	\begin{bmatrix} b_1 & b_2 \\ b_3 & b_4 \\ \end{bmatrix}
	=
	\begin{bmatrix} a_1 b_1 + a_2 b_3 & a_1 b_2 + a_2 b_4 \\ a_3 b_1 + a_4 b_3 & a_3 b_2 + a_4 b_4 \\ \end{bmatrix} \\ \label{eq:matrix2}
	det(C) = &
	det(AB) = det(A) \cdot det(B) \neq 0
\end{align}

Then, since $C$ has real entries as demonstrated in equation \ref{eq:matrix} and since it is nonsingular as demonstrated by the nonzero determinant in equation \ref{eq:matrix2}, we conclude that $C \in G$ and that $G$ is closed under $*$.

From this point on, $A, B$ and $C$ refer to nonasingular 2 by 2 matrices with arbitrary entries and no specific relationship to each other.

Consider the following arithmetic:


\begin{align}
	\label{eq:matrix3}
	A * B = &
	\begin{bmatrix} a_1 & a_2 \\ a_3 & a_4 \\ \end{bmatrix}
	\cdot
	\begin{bmatrix} b_1 & b_2 \\ b_3 & b_4 \\ \end{bmatrix}
	=
	\begin{bmatrix} a_1 b_1 + a_2 b_3 & a_1 b_2 + a_2 b_4 \\ a_3 b_1 + a_4 b_3 & a_3 b_2 + a_4 b_4 \\ \end{bmatrix}
	\\ \label{eq:matrix4}
	B * C = &
	\begin{bmatrix} b_1 & b_2 \\ b_3 & b_4 \\ \end{bmatrix}
	\cdot
	\begin{bmatrix} c_1 & c_2 \\ c_3 & c_4 \\ \end{bmatrix}
	=
	\begin{bmatrix} b_1 c_1 + b_2 c_3 & b_1 c_2 + b_2 c_4 \\ b_3 c_1 + b_4 c_3 & b_3 c_2 + b_4 c_4 \\ \end{bmatrix}
	\\ \label{eq:matrix5}
	A * (B * C) = &
	\begin{bmatrix} a_1 & a_2 \\ a_3 & a_4 \\ \end{bmatrix}
	\cdot
	\begin{bmatrix} b_1 c_1 + b_2 c_3 & b_1 c_2 + b_2 c_4 \\ b_3 c_1 + b_4 c_3 & b_3 c_2 + b_4 c_4 \\ \end{bmatrix}
	\\ = & 
	\begin{bmatrix}
		a_1 b_1 c_1 + a_1 b_2 c_3 + a_2 b_3 c_1 + a_2 b_4 c_3 &
		a_1 b_1 c_2 + a_1 b_2 c_4 + a_2 b_3 c_2 + a_2 b_4 c_4 \\
		a_3 b_1 c_1 + a_3 b_2 c_3 + a_4 b_3 c_1 + a_4 b_4 c_3 &
		a_3 b_1 c_2 + a_3 b_2 c_4 + a_4 b_3 c_2 + a_4 b_4 c_4 \\
	\end{bmatrix}
	\\ \label{eq:matrix6}
	(A * B) * C = &
	\begin{bmatrix} a_1 b_1 + a_2 b_3 & a_1 b_2 + a_2 b_4 \\ a_3 b_1 + a_4 b_3 & a_3 b_2 + a_4 b_4 \\ \end{bmatrix}
	\cdot
	\begin{bmatrix} c_1 & c_2 \\ c_3 & c_4 \\ \end{bmatrix}
	\\ = & 
	\begin{bmatrix}
		a_1 b_1 c_1 + a_1 b_2 c_3 + a_2 b_3 c_1 + a_2 b_4 c_3 &
		a_1 b_1 c_2 + a_1 b_2 c_4 + a_2 b_3 c_2 + a_2 b_4 c_4 \\
		a_3 b_1 c_1 + a_3 b_2 c_3 + a_4 b_3 c_1 + a_4 b_4 c_3 &
		a_3 b_1 c_2 + a_3 b_2 c_4 + a_4 b_3 c_2 + a_4 b_4 c_4 \\
	\end{bmatrix}
\end{align}

By equations \ref{eq:matrix5} and \ref{eq:matrix6}, we have that $G$ is associative under $*$.

Consider the following:



\begin{align} \label{eq:id}
	A * I =
	\begin{bmatrix} a_1 & a_2 \\ a_3 & a_4 \\ \end{bmatrix}
	\cdot
	\begin{bmatrix} 1 & 0 \\ 0 & 1 \\ \end{bmatrix}
	=
	\begin{bmatrix} a_1 & a_2 \\ a_3 & a_4 \\ \end{bmatrix}
	= A =
	\begin{bmatrix} 1 & 0 \\ 0 & 1 \\ \end{bmatrix}
	\cdot
	\begin{bmatrix} a_1 & a_2 \\ a_3 & a_4 \\ \end{bmatrix}
	= I * A
\end{align}

	Then, $I = \begin{bmatrix} 1 & 0 \\ 0 & 1 \end{bmatrix}$ is the identity element of $G$.

	
	Let us write $A^{-1}$ to mean $\ainverse$
	Consider one last set of equations:

\begin{align}
	\label{eq:inv1}
	A * A^{-1} =
	\begin{bmatrix} a_1 & a_2 \\ a_3 & a_4 \\ \end{bmatrix}
	\cdot
	\ainverse
	=
\begin{bmatrix}
	\frac{a_1 a_4 -a_2 a_3 }{a_1 a_4 -a_2 a_3 } & \frac{a_2 a_4 -a_2 a_4 }{a_1 a_4 -a_2 a_3 } \\
	\frac{-a_1 a_3 +a_1 a_3 }{a_1 a_4 -a_2 a_3 } & \frac{-a_2 a_3 +a_1 a_4 }{a_1 a_4  - a_2 a_3 } \\
\end{bmatrix}
	\begin{bmatrix} 1 & 0 \\ 0 & 1 \\ \end{bmatrix}
	= I \\
	\label{eq:inv2}
	A^{-1} * A =
	\ainverse
	\cdot
	\begin{bmatrix} a_1 & a_2 \\ a_3 & a_4 \\ \end{bmatrix}
	=
\begin{bmatrix}
	\frac{a_1 a_4 -a_2 a_3 }{a_1 a_4 -a_2 a_3 } & \frac{a_2 a_4 -a_2 a_4 }{a_1 a_4 -a_2 a_3 } \\
	\frac{-a_1 a_3 +a_1 a_3 }{a_1 a_4 -a_2 a_3 } & \frac{-a_2 a_3 +a_1 a_4 }{a_1 a_4  - a_2 a_3 } \\
\end{bmatrix}
	\begin{bmatrix} 1 & 0 \\ 0 & 1 \\ \end{bmatrix}
	= I
\end{align}

Then, any element $a \in G$ has an inverse $\ainverse \in G$.

Because $G$ is closed under $*$, is associative, has an identity element, and has an inverse for each element,$\langle G, * \rangle$ is a group.
	

\end{proof}

\begin{thm} \label{abel}
	$\langle G, * \rangle $ is not an Abelian group.
\end{thm}

\begin{proof}
	Let $A = \begin{bmatrix} 1 & 1 \\ -1 & 1 \end{bmatrix}$
	and $B = \begin{bmatrix} 1 & 0 \\ -1 & 1 \end{bmatrix}$.
	Notice that $AB = \begin{bmatrix} 0 & 1 \\ -2 & 1 \end{bmatrix}$
	while $BA = \begin{bmatrix} 1 & 1 \\ -2 & 0 \end{bmatrix}$.
	Then, we have some $A,B \in G$ where $AB \neq BA$.
	We conclude that $G$ is not commutative under $*$
	and by definition $G$ is not an Abelian group.
\end{proof}

\section{A subset of nonsingular 2 $\times$ 2 matrices}


Let $G$ be defined as follows:
\begin{align*}
	G = \{A = \begin{bmatrix} a & b \\ c & d \\ \end{bmatrix} \in \reals^{2 \times 2} &
	\ni \exists A^{-1} \in \reals^{2 \times 2} \ni AA^{-1} = A^{-1}A = \begin{bmatrix} 1 & 0 \\ 0 & 1 \\ \end{bmatrix}
\\ & \land a + c = 1 \land b + d = 1\}
\end{align*}

Let $*$ be defined on $G$ as standard matrix multiplication.

\begin{thm} \label{thm:unitcols}
	$\langle G, * \rangle$ is a group.
\end{thm}

\begin{proof}
	Let the definitions in equation \ref{eq:matrix_defs} hold and suppose $C = AB$ as described by equation \ref{eq:matrix}. We have the following equations by taking the sum of the colunms of $C$ from equation \ref{eq:matrix}:
	\begin{align}
		a_1 + a_3 = a_2 + a_4 = b_1 + b_3 + b_2 + b_4 = 1 \\
		a_1 b_1 + a_2 b_3 + a_3 b_1 + a_4 b_3 = b_1(a_1+a_3) + b_3(a_2+a_4) = b_1 + b_3 = 1 \\
		a_1 b_2 + a_2 b_4 + a_3 b_2 + a_4 b_4 = b_2(a_1+a_3) + b_4(a_2+a_4) = b_2 + b_4 = 1
	\end{align}
	Since the columns of $C$ sum to $1$, we have that $C \in G$ and in general that $G$ is closed under $*$.

	Now, suppose $A,B$ and $C$ are defined with generic real entries as described by equation \ref{eq:matrix_defs}.

By equations \ref{eq:matrix5} and \ref{eq:matrix6}, we have that $G$ is associative under $*$, since those equations hold for our current defintiion of $G$.

By the same reasoning demonstrated in equation \ref{eq:id}, we have that $I = \begin{bmatrix} 1 & 0 \\ 0 & 1 \end{bmatrix}$ is the identity for $G$.

By the same reasoning demonstrated in equations \ref{eq:inv1} and \ref{eq:inv2}, we have that $\ainverse$ is the inverse of any $A = \begin{bmatrix} a_1 & a_2 \\ a_3 & a_4 \end{bmatrix} \in G$.

Then, since $G$ is closed under $*$, since $G$ is associative under $*$, since $G$ has an identity element and since every element of $G$ has an inverse, we conclude that $\langle G, * \rangle$ is a group.

\end{proof}

\pagebreak
\end{document}
