\documentclass[12pt]{article}
\usepackage{url,amsmath,amsthm,enumitem,amsfonts,tikz,verbatim,amssymb, wasysym,multicol}
\usepackage[makeroom]{cancel}
\usetikzlibrary{arrows}

\title{Abstract Algebra: Homework \#1}
\date{Wednesday 27 May 2020}
\author{Joel Savitz}

\newcommand{\reals}{\mathbb{R}}
\newcommand{\ints}{\mathbb{Z}}
\newtheorem{thm}{Theorem}

\begin{document}

\maketitle

\section{Chapter 3: Excercise A1}

Suppose $*$ is defined on $\reals$ as $a * b = a + b + k$ for any $a,b \in \reals$ for some $k \in \reals$.

\begin{thm}
	$\langle \reals, * \rangle$ is a group.
\end{thm}

\begin{proof}
	Let $a,b,$ and $c$ be some arbitrary real numbers.
	Because $a * b = a + b + k \in \reals$, we have that the real numbers are closed under $*$.
	Then, observe that $a * (b * c)  = a * (b + c + k) = a + (b + c + k) + k = (a + b + k) + c + k = (a + b + k) * c = (a * b) * c$, so $*$ is associative.
	We also have $a * -k = a + (-k) + k = -k + a + k = -k * a = a$, so $-k$ is the identity for the real numbers under $*$
	Finally, consider the quantity $-(2k+a)$. Since we have $a * -(2k+a) = a + (-2k) +  (-a) + k = -(2k+a) * a = -k$, that quantity is the inverse of any $a$.
	Since $*$ is closed under the real numbers and $*$ is associative and $-k$ is the identity of $*$ under the real numbers and any real number $a$ has an inverse under $*$ of $-(2k-a)$, $\langle \reals, * \rangle$ is a group.
\end{proof}

\section{Chapter 3: Excercise A3}

Suppose $*$ is defined on $\reals$ as $a * b = a + b ab$ for any $a,b \in \reals$.

\begin{thm}
	$\langle \reals, * \rangle$ is a group.
\end{thm}

\begin{proof}
	Let $a,b,$ and $c$ be some arbitrary real numbers.
	Because $a * b = a + b + ab \in \reals$, we have that the real numbers are closed under $*$.
	Then, observe that $a * (b * c)  = a * (b + c + bc) = a + b + c + ab + ac + bc + abc = (a + b + ab) * c = (a * b) * c$, so $*$ is associative.
	We also have $a * 0 = a + (0) + 0a = 0 * a = a$, so $0$ is the identity for the real numbers under $*$
	Finally, consider the quantity $\frac{-a}{1+a}$.
	Since we have $a * \frac{-a}{1+a} = a + \frac{-a}{1+a} +  \frac{-a^2}{1+a} = \frac{-a}{1 + a} * a = \frac{a^2+a}{1+a} + \frac{-a}{1+a} + \frac{-a^2}{1+a} = 0$, that quantity is the inverse of any $a$.
	Since $*$ is closed under the real numbers and $*$ is associative and $0$ is the identity of $*$ under the real numbers and any real number $a$ has an inverse under $*$ of $\frac{-a}{1+a}$, $\langle \reals, * \rangle$ is a group.
\end{proof}

\section{Chapter 3: Excercise B1}

Suppose $*$ is defined on $\reals \times \reals$ as $(a,b) * (c,d) = (ad+bc,bd)$ for any $(a,b),(c,d) \in \reals \times \reals$.

\begin{thm}
	$\langle \reals, * \rangle$ is a group.
\end{thm}

\begin{proof}
	Let $(a,b),(c,d),(e,f) \in \reals \times \reals$.
	Consider that $(a,b) * (c,d) = (ad+bc,bd) \in \reals \times \reals$ since $ad+bc \in \reals \land bc \in \reals$. Then, $\reals \times \reals$ is closed under $*$
	Now observe the following equivalence: $\Big((a,b) * (c,d)\Big) * (e,f) = (ad+bc,bd) * (e,f) = (adf+bcf+bde,bdf) = (a,b) * (cf+de,df) = (a,b) * \Big((c,d) * (e,f)\Big)$
	Since $(a,b) * (0,1) = (1a+0b,1b) = (a,b) = (0b + 1a,1b) = (0,1) * (a,b)$, we have that $(0,1)$ is the identity of the real numbers under $*$,
	Because we have $(a,b)*(\frac{-a}{b^2},\frac{1}{b})=(\frac{a}{b}+\frac{-a}{b}, b \cdot \frac{1}{b} = (0,1) = (\frac{-a}{b}+\frac{a}{b},\frac{1}{b}) = (\frac{-a}{b^2},\frac{1}{b}) * (a,b)$, that pair $(\frac{-a}{b^2},\frac{1}{b})$ is the inverse of any real $a$ under $8$.
	Then, since $*$ is a closed associative operation on the real numbers with an identity and an inverse for any element of the real numbers, we have that $\langle \reals, * \rangle$ is a group.
	
\end{proof}

\end{document}
