\documentclass[12pt]{article}
\usepackage{amsmath, amsthm, amsfonts, graphicx}

\usepackage{euler}
\title{Abstract Algebra: Homework \#9}
\author{Joel Savitz}
\date{Wednesday 22 July 2020}

\newcommand{\nats}{\mathbb{N}}
\newcommand{\reals}{\mathbb{R}}
\newcommand{\rats}{\mathbb{Q}}
\newcommand{\ints}{\mathbb{Z}}
\newcommand{\gltwo}{GL_2(\reals)}
\newcommand{\sltwo}{SL_2(\reals)}
\newcommand{\glmatrix}[4]{\ensuremath{\begin{bmatrix} #1 & #2 \\ #3 & #4 \end{bmatrix}}}
\newcommand{\glxmatrix}[9]{\ensuremath{\begin{bmatrix} #1 & #2 & #3 \\
#4 & #5 & #6 \\ #7 & #8 & #9 \end{bmatrix}}}
\newcommand{\glinverse}[4]{\ensuremath{\frac{1}{#1 #4 - #2 #3}\glmatrix{#4}{-#2}{-#3}{#1}}}
\newcommand{\ord}{\operatorname{ord}}
\newcommand{\Aut}{\operatorname{Aut}}
\newcommand{\freals}{\mathcal{F}(\reals)}
\newcommand{\dreals}{\mathcal{D}(\reals)}
\newcommand{\creals}{\mathcal{C}(\reals)}
\newcommand{\powset}{\mathcal{P}}
\newtheorem{thm}{Theorem}
\newtheorem{cnt}{Counterexample}

\begin{document}
\maketitle

\section{A consequence of the fundamental homomorphism theorem}

Suppose $G = \ints_{15}$ and $H = \ints_5$.

Let $f: G \to H \ni f = \binom{0\ 1\ 2\ 3\ 4\ 5\ 6\ 7\ 8\ 9\ 10\ 11\ 12\ 13\ 14}{\; 0\ 1\ 2\ 3\ 4\ 0\ 1\ 2\ 3\ 4\ 0\ \; \; 1\ \; \; 2\ \; \; 3\ \; \; 4\ }$ be a hommorphism from $G$ to $H$.

We have $\ker(f) = \{x \in G: f(x) = 0 \in H \} = \{0, 5, 10\} \subseteq G$,
therefore $G/H = \{ \{0, 5, 10\}, \{1, 6, 11\}, \{2, 7, 12\}, \{3, 8, 13\}, \{4, 9, 14\} \}$.

By the fundamental homomorphism theorem, we have $H \cong G/K$ since $K$ is the kernel of a homomorphism from $G$ to $H$.

\section{Chapter 16, Exercise A3}

Suppose $G = S_3$ and $H = \ints_2$.

Denote the elements of $S_3$ as follows:
\begin{align}
	\label{topperm}
	r_1 = & \binom{1\ 2\ 3}{2\ 3\ 1} &
	r_2 = & \binom{1\ 2\ 3}{3\ 1\ 2} &
	r_3 = & \binom{1\ 2\ 3}{1\ 2\ 3} \\
	\label{botperm}
	f_1 = & \binom{1\ 2\ 3}{1\ 3\ 2} &
	f_2 = & \binom{1\ 2\ 3}{3\ 2\ 1} &
	f_3 = & \binom{1\ 2\ 3}{2\ 1\ 3}
\end{align}

Then, table \ref{t1} describes the behavior of $G$ under function composition.

\begin{table}[!ht] 
\begin{tabular}{l|llllll}
	$\circ$ & $r_3$ & $r_1$ & $r_2$ & $f_1$ & $f_2$ & $f_3$	\\ \hline
	$r_3$ & $r_3$ & $r_1$ & $r_2$ & $f_1$ & $f_2$ & $f_3$   \\
	$r_1$ & $r_1$ & $r_2$ & $r_3$ & $f_3$ & $f_1$ & $f_2$ 	\\
	$r_2$ & $r_2$ & $r_3$ & $r_1$ & $f_2$ & $f_3$ & $f_1$ 	\\
	$f_1$ & $f_1$ & $f_3$ & $f_2$ & $r_3$ & $r_2$ & $r_1$ 	\\
	$f_2$ & $f_2$ & $f_1$ & $f_3$ & $r_1$ & $r_3$ & $r_2$ 	\\
	$f_3$ & $f_3$ & $f_2$ & $f_1$ & $r_2$ & $r_1$ & $r_3$	\\
\end{tabular}
\centering
\caption{Operation table for $G$ under $\circ$}
\label{t1}
\end{table}

Define $K = \{ r_1, r_3, r_3\}$

\begin{thm} \label{thm1}
	$H \cong G/K$
\end{thm}

\begin{proof}
	Define $\phi:G \to H$
	by $f = \{ (r_1, 0), (r_2, 0), (r_3,0), (f_1,1),(f_2,1),(f_3,1) \}$.
	By inspection, $f$ is surjective, since by definition it maps some element of $G$ to
	both $0 \in H$ and $1 \in H$ at least once, so it is surjective.
	Let $x,y \in S_3$ and consider $f(x) + f(y)$.
	We will use the term rotation to refer any $x \in G$
	where $f(x) = 0$ and flip to refer to any $x \in G$
	where $f(x) = 1$. Since $f$ maps every element of $G$ to an element of $H$,
	and $H = \{ 0, 1\}$, every element of $G$ is either a flip or a rotation.
	Suppose $x$ is a flip and $y$ is a flip. Then, $xy$ is a rotation by
	table \ref{t1} so $f(xy) = 0$ and $f(x) + f(y) = 1 + 1 = 0$,
	thus $f(xy) = f(x) + f(y)$.
	Suppose $x$ is a rotation and $y$ is a rotation. Then, $xy$ is a rotation
	by table \ref{t1} so $f(xy) = 0$ and $f(x) + f(y) = 0 + 0 = 0$,
	thus $f(xy) = f(x) + f(y)$.
	Suppose $x$ is a flip and $y$ is a rotation. Then, by table \ref{t1},
	$xy$ is a flip. Alternatively, if $x$ is a rotation and $y$ is a flip,
	table \ref{t1} still constrains us such that $xy$ must be a flip,
	thus we will consider the two cases together. Then,
	we have either $f(x) = 0 \land f(y) = 1$ or $f(x) = 1 \land f(y) = 0$,
	but in both cases we have $f(x) + f(y) = 1 = f(xy)$.
	Since the proposition holds for all possible cases,
	we have that for any $x,y \in G$, it holds that $f(x) + f(y) = f(xy)$,
	so we conclude that $f$ is a homomorphism onto $H$.

	By inspection, $\ker(f) = K = \{r_1, r_2, r_3\}$ since this is
	exactly the subset of $G$ containing every element that is mapped
	to $0 \in H$ by $f$. As demonstrated above, every rotation composed
	with a rotation is simply another rotation,
	and $r_2 \circ r_1 = r_1 \circ r_2 = r_3 \circ r_3 = r_3$,
	so $K$ is a subgroup of $G$, and since it is the kernel of a homomorphism,
	$K$ is furthermore a normal subgroup of $G$, so $G/K$ is well defined.
	By the fundamental homomorphism theorem, the quotient group formed by
	the kernel of a homomorphism onto the codomain
	is isomorphic to the range of the homomorphism, so $H \cong G/K$.
	This proves theorem \ref{thm1}.
\end{proof}

Table \ref{t2} gives the operation table for $H$ under $\circ$.

\begin{table}[!ht] 
\begin{tabular}{l|lll}
	$\circ$ & $r_3$ & $r_1$ & $r_2$ \\ \hline
	$r_3$ & $r_3$ & $r_1$ & $r_2$ \\
	$r_1$ & $r_1$ & $r_2$ & $r_3$ \\
	$r_2$ & $r_2$ & $r_3$ & $r_1$
\end{tabular}
\centering
\caption{Operation table for $H$ under $\circ$}
\label{t2}
\end{table}


Table \ref{t3} gives the operation table for $G/K$ under coset composition.

\begin{table}[!ht] 
\begin{tabular}{l|lll}
	$\circ$ & $K$ & $Kr_1$ & $Kr_2$ \\ \hline
	$K$ & $K$ & $Kr_1$ & $Kr_2$ \\
	$Kr_1$ & $Kr_1$ & $Kr_2$ & $K$ \\
	$Kr_2$ & $Kr_2$ & $K$ & $Kr_1$
\end{tabular}
\centering
\caption{Operation table for $G/K$ under coset $\circ$}
\label{t3}
\end{table}

\section{Chapter 16, Exercise A4}

Suppose $a, b,$ and $c$ are some sets.

Let $G = \powset(\{a,b,c\})$, $H = \powset(\{a,b\})$, and $K = \{ \emptyset, \{ c \}\}$.

\begin{thm} \label{thm2}
	$H \cong G/K$
\end{thm}

\begin{proof}
	Define $f:G \to H$ by $f(X) = X \cap \{ a,b\} \in H \ni X \in G$
	and assume it is a homomorphism from $G$ to $H$.
	Let $z \in H$. Clearly $f(z) = z$, so $f$ is surjective.
	Then, suppose $x \in G \ni f(x) = \emptyset$.
	Since $f(x) = x \cap \{a,b\}$, we have that $f(x) = \emptyset$
	only if $a \not\in x \land b \not\in x$ by definition of set intersection.
	Then, we construct this set by the axiom schema of comprehension over
	this last predicate, i.e. we have $\ker(f)  = \{x \in \powset(G): a \not\in x \land b \not\in x\}$.
	Suppose $y \in K$. By inspection $a \not\in y \land b \not\in y$.
	Thus, $K \subseteq \ker(f)$. Alternatively take some $y \in \ker(f)$.
	By the definition of this set, $a \not\in y \land b \not\in y$, and since
	there are no other elements in $\powset(G)$ besides $\{c\}$ and $\emptyset$,
	we must have $y \in K$, so $\ker(f) \subseteq K$, and by the axiom of extentionality,
	we have $K \subseteq \ker(f) \land \ker(f) \subseteq K \iff K = \ker(f)$,
	and by the fundamental homomorphism theorem, the quotient group formed by the kernel
	of a homomorphism onto the codmain is isomorphic to the range of that homormorphism,
	so $H \cong G/K$ and this proves theorem \ref{thm2}.
\end{proof}

\section{Chapter 16, Exercise A5}

Suppose $G = \ints_3 \times \ints_3$ and $H = \ints_3$ and $K = \{ (0,0), (1,1), (2,2) \}$.


\begin{thm} \label{thm3}
	$H \cong G/K$
\end{thm}

\begin{proof}
	Define $f:G \to H$ by $f(a,b) = a - b$
	and assume $f$ is a homomorphism.
	Since for all $a,b \in \ints_3$,
	we have $a - b = 0 \iff a = b$,
	we must have that $\ker(f) = K$ since these
	are all the elements of $G$ of the form $(a,a) \ni a \in \ints_3$.
	To verify this, inspect $K$.
	Suppose $x \in \ints_3$.
	Then at the very least, we have $f((x,0)) = x \ni (x,0) \in \ints_3 \times \ints_3$,
	so $f$ is surjective.
	Then, since by the fundamental homomorphism theorem 
	the group formed by the kernel of a homomorphism
	onto the codomain
	is isomorphic to the range of the homomorphism,
	we have $H \cong G/K$ and this proves theorem \ref{thm3}.
\end{proof}

\section{The FHT applied to $\freals$}

Define $\alpha: \freals \to \reals$ by $\alpha(f) = f(1) \ni f \in \freals$,
and define $\beta: \freals \to \reals$ by $\beta(f) = f(2) \ni f \in \freals$.

Instead of using the normal, boring terminology of ``surjective homomorphism'',
I will use an equivalent and much cooler term from category theory,
the epic morphism, or epimorphism. To be precise for this context, an epic morphism is
a surjective homomorphism.

\begin{thm} \label{thm4}
	$\alpha$ and $\beta$ are epic morphisms from $\freals \to \reals$.
\end{thm}

\begin{proof}
	Suppose $f,g \in \freals$.
	Then, $\alpha(f) + \alpha(g) = f(1) + g(1)  = (f + g)(1) = \alpha(f + g)$
	and $\beta(f) + \beta(g) = f(2) + g(2) = (f + g)(2) = \beta(f + g)$
	since the vector addition of real valued functions is linear,
	so $\alpha$ and $\beta$ are homomorphisms.
	To verify the epic property of these morphisms,
	take $a \in \reals$,
	and define $f \in \freals \ni f(x) = (x - 1) + a$
	and define $g \in \freals \ni g(x) = (x - 2) + a$.
	By inspection, we see that $\alpha(f) = a$ and $\beta(g) = a$,
	so $\alpha$ and $\beta$ are surjetive,
	i.e. $\alpha$ and $\beta$ are epic morphisms from $\freals \to \reals$
	and this proves theorem \ref{thm4}.
\end{proof}


Now suppose $J = \{ f \in \freals: (1,0) \in f \}$ and
$K = \{ f \in \freals: (2,0) \in f \}$.

\begin{thm} \label{thm5}
	$\reals \cong \freals/J \land \reals \cong \freals/K$
\end{thm}

\begin{proof}
	Suppose $f \in \freals$.
	Since $\alpha(f) = f(1) = 0$ if only if
	$(1,0) \in f$, and this is exactly the predicate
	of the set comprehension used to construct $J$,
	we have $\ker(\alpha) = J$.
	Since $\beta(f) = f(2) = 0$ if only if
	$(2,0) \in f$, and this is exactly the predicate
	of the set comprehension used to construct $K$,
	we have $\ker(\beta) = K$.
	By theorem \ref{thm4}, we have that
	$\alpha$ and $\beta$ are epic morphisms,
	so by the fundamental homomorphism theorem,
	the quotient groups formed by the kernel of
	some epic morphism are isomorphic to the range
	of that morphism, that is to say,
	$\reals \cong \freals/J \land \reals \cong G/K$
	and this proves theorem \ref{thm5}.
\end{proof}

\begin{thm} \label{thm6}
	$\freals/J \cong \freals/K$.
\end{thm}

\begin{proof}
	$\reals \cong \freals/J \land \reals \cong \freals/K \iff \freals/J \cong \freals/K$
	by the transitivity of the equivalence relation that is group isomorphism.
	This proves theorem \ref{thm6}. lol.
\end{proof}

\section{Chapter 16, Exercise C1}

Suppose $G$ is some abelian group,
and let $H = \{x \in G: x = y^2 \ni y \in G \}$.

\begin{thm} \label{thm7}
	The function $f:G \to H \ni f(x) = x^2 \ni x \in G$ is an epic morphism from $G$ to $H$.
\end{thm}

\begin{proof}
	Suppose $x,y \in G$.
	Since $G$ is abelian, we have $f(x)f(y) = xxyy = xyxy = f(xy)$,
	so the status of $f$ as a homomorphism is established.
	Then, consider any $y \in H$. By definition of $H$,
	there must exist some $x \in G$ such that $x^2 = G$,
	so $f(x) = x^2 = y \in H$ and $f$ is an epic morphism.
	This proves theorem \ref{thm7}.
\end{proof}

\section{Chapter 16, Exercise F}

For any two subgroups $A$ and $B$ of some group $G$,
define $AB = \{x \in G: x = yz \ni y \in A \land z \in B \}$.

\begin{thm} \label{thm9}
	For any group $G$ with a normal subgroup $H$ and a subgroup $K$,
	we have $K/(H \cap K) \cong HK/H$.
\end{thm}

\begin{proof}
	Since $H \trianglelefteq G$,
	we must have that for any $x \in H$,
	and any $a \in G$
	we have $axa^{-1} \in H$.
	If we take some $b \in K$,
	we must also then have $bxb^{-1} \in H$
	since $K \subseteq G$.
	Every element of $H \cap K$ has it's inverse
	in  $H \cap K$ if and only if this is true
	for both $H$ and $K$ individually,
	which it is by definition of a subgroup,
	and since $H \cap K \subseteq K$ by definition of set intersection,
	we have that $(H \cap K) \trianglelefteq K$.

	Now we consider the set $HK$.
	Trivially, $HK \subseteq G$ since groups are closed under their operation.
	Let $a,b \in HK$ then, $a = rs \ni r \in H \land s \in K$
	and $b = jk \ni j \in H \land k \in K$.
	Therefore $ab = (rs)(jk)$, and since $H$ is normal with respect to $G$,
	it commutes with every element of $G$ and the elements of $K$ are among them
	due to subset status. So, we can rearange $ab = (rj)(sk) \in HK$
	since $rj \in H$ and $sk \in K$. Therefore $HK$ is closed under the group operation.
	Suppose $x = yz \in HK \ni y \in H \land z \in K$.
	Then, $y^{-1} \in H \land z^{-1} \in K$,
	and since $e \in H \land e \in K$, we have $y^{-1}e \in HK \land ez^{-1} \in K$,
	and of course $z^{-1}y^{-1} = (yz)^{-1} \in HK$, so every element of $HK$
	has its inverse in $HK$, so $HK$ is a subgroup of $G$.
	Now since every element $x$ of $H$ can be written as $xe \ni e \in K$,
	it is clear that $x \in HK$, so $H \subseteq HK$.
	Since $H$ is a subgroup of $G$, every $a,b \in H$ satisfies $ab \in H$,
	so this also applies with respect to $K$ and we additionally conclude
	from the fact that $H$ is already a subgroup that every element of $H$
	has its inverse in $H$, so $H \leq K$.
	Since $HK \subseteq G$, it is trivial that for any $x \in H$,
	we have $axa^{-1} \in H$ for any $a \in HK$,
	so in fact we have $H \trianglelefteq HK$.
	With this last fact established,
	we have that $HK/H$ must be well-defined.

	Take some $x \in HK/H$.
	By definition, this $x$ can be written as $x = Ha \ni a \in HK$,
	and then any $y \in Ha$ can be written as $y = ha \ni a \in HK$,
	and then this $y$ can be written as some $pq \ni p \in H \land q \in K$,
	therefore $y = hpq$. Since $h \in H \land p \in H$, we must have $hp \in H$.
	We also have $q \in K$. Since $y = (hp)q \in Ha$ we have that any $Ha$
	can be written as $Hq$ for some $q \in K$.

	Now define the function $f:K \to HK/H \ni f(k) = Hk \ni k \in K$.
	Let $a,b \in K$. Then, we see that $f(a) f(b) = (Ka)(Kb) = K(ab) = f(ab)$
	so $f$ is a homomorphism. Consider any $y \in HK/H$.
	Since $y$ can be written as some $Hk \ni k \in K$,
	we must have a $k \in K$ where $f(k) = Ka$ therefore $f$ is surjective
	and is an epic morphism from $K$ to $HK/H$.

	Suppose $x \in H \cap K$.
	Then, $f(x) = Hx = H$ since $x \in H$.
	Therefore $x \in \ker(f)$.
	Alternatively, suppose $x \in \ker(f)$.
	Then, $f(x) = Hx = H$, so $x \in H \cap K$ or else this would not hold,
	and by the axiom of extentionality,
	we have $H \cap K \subseteq \ker(f) \land \ker(f) \subseteq H \cap K \iff \ker(f) = H \cap K$.
	By the fundamental homomorphism theorem,
	we have that the quotient group formed by
	the kernel of an epic morphism is isomorphic to the range of the morphism,
	i.e. $H \cap K \cong HK/H$.
	This proves theorem \ref{thm9}.
\end{proof}

\section{Chapter 16, Exercise I}

\begin{thm} \label{thm8}
	For any two normal subgroups $H$ and $K$ of some group $G$ where $H \subseteq K$
	and some function $\phi:G/H \to G/K \ni \phi(Ha) = Ka \ni Ha \in G/H$,
	we have that $(G/H)/(K/H) \cong G/K$.
\end{thm}

\begin{proof}
	Suppose $Ha, Hb \in G/H \ni Ha = Hb$.
	Then $\phi(Ha) = Ka$ and $\phi(Hb) = Kb$,
	but since $Ha = Hb \iff ab^{-1} \in H$,
	we must have $Ka = Kb \iff ab^{-1} \in K$
	since $H \subseteq K$.
	Therefore $\phi$ is well-defined.
	We see that $\phi(Ha)\phi(Hb) = (Ka)(Kb) = K(ab) = \phi(H(ab)) = \phi((Ha)(Hb))$,
	so $\phi$ must be a homomorphism.
	Furthermore, for any $Ka \in G/K$, it is obvious from the definition that $\phi(Ha) = Ka$,
	so $\phi$ maps $G/H$ onto $G/K$, so $\phi$ is an epic morphism.
	Consider an element $x \in K/H = \{x \in G/H: x = Ha \ni a \in K\}$.
	Since the identity element of $G/K$ is nothing but $K$ itself,
	we see that $\phi(x) = K$ since $x = Ha \ni a \in K$ and $K = Ka$ for any $a \in K$.
	therefore, $K/H \subseteq \ker(\phi)$.
	Alternatively, consider some $y \in \ker(\phi)$.
	Since $\phi(y) = K = Ka \ni a \in K$,
	we must have that $y$ is of the form $y = Ha \ni a \in K$,
	therefore $y \in K/H$ by the definition of quotient group
	and $\ker(\phi) \subseteq K/H$ and then by the axiom of extentionality,
	$\ker(\phi) = K/H$.
	Finally, since $K/H$ is the kernel of an epic morphism from $G/H$
	to $G/K$, we must have $(G/H)/(K/H) \cong G/K$
	by the fundamental homomorphism theorem.
	This proves theorem \ref{thm8}.
\end{proof}

\end{document}
