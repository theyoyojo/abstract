\documentclass[12pt]{article}
\usepackage{amsmath, amsthm, amsfonts, graphicx}

\usepackage{euler}
\title{Abstract Algebra: Homework \#8}
\author{Joel Savitz}
\date{Wednesday 15 July 2020}

\newcommand{\nats}{\mathbb{N}}
\newcommand{\reals}{\mathbb{R}}
\newcommand{\rats}{\mathbb{Q}}
\newcommand{\ints}{\mathbb{Z}}
\newcommand{\gltwo}{GL_2(\reals)}
\newcommand{\sltwo}{SL_2(\reals)}
\newcommand{\glmatrix}[4]{\ensuremath{\begin{bmatrix} #1 & #2 \\ #3 & #4 \end{bmatrix}}}
\newcommand{\glxmatrix}[9]{\ensuremath{\begin{bmatrix} #1 & #2 & #3 \\
#4 & #5 & #6 \\ #7 & #8 & #9 \end{bmatrix}}}
\newcommand{\glinverse}[4]{\ensuremath{\frac{1}{#1 #4 - #2 #3}\glmatrix{#4}{-#2}{-#3}{#1}}}
\newcommand{\ord}{\operatorname{ord}}
\newcommand{\Aut}{\operatorname{Aut}}
\newcommand{\freals}{\mathcal{F}(\reals)}
\newcommand{\dreals}{\mathcal{D}(\reals)}
\newcommand{\creals}{\mathcal{C}(\reals)}
\newtheorem{thm}{Theorem}
\newtheorem{cnt}{Counterexample}

\begin{document}
\maketitle

\section{Chapter 15, Exercise A1}

Suppose $G = \ints_{10} \land H = \{ 0, 5 \}$.

Then, table \ref{t1} describes the operation table for $G/H$ with respect to coset multiplication defined for cosets of an abelian group, denoted $*$.

I exclusively use multiplicative notation here because I like it better, but $aH \ni a \in G$ denotes the coset $a +_{10} H$.
Since $G$ is abelian, I use left and right cosets interchangably.

The following are the elements of $G/H$:

\begin{align*}
	H0 = & \{ 0, 5 \} \\
	H1 = & \{ 1, 6 \} \\
	H2 = & \{ 2, 7 \} \\
	H3 = & \{ 3, 8 \} \\
	H4 = & \{ 4, 9 \}
\end{align*}

\begin{table}[!ht] 
\begin{tabular}{l|lllll}
	$*$ & $H0$ & $H1$ & $H2$ & $H3$ & $H4$	\\ \hline
	$H0$ & $H0$ & $H1$ & $H2$ & $H3$ & $H4$	\\
	$H1$ & $H1$ & $H2$ & $H3$ & $H4$ & $H0$	\\
	$H2$ & $H2$ & $H3$ & $H4$ & $H0$ & $H1$	\\
	$H3$ & $H3$ & $H4$ & $H0$ & $H1$ & $H2$	\\
	$H4$ & $H4$ & $H0$ & $H1$ & $H2$ & $H3$	\\
\end{tabular}
\centering
\caption{Operation table for $G/H$ under $*$}
\label{t1}
\end{table}

If we replace each $HX$ in the table with an $f(HX)$ where $f:G/H \to \ints_5 \ni f(HX) = X$ and replace $*$ by $+_5$,
we construct the operation table for $\ints_5$. By table inspection, this $f$ is an isomorphism from $G/H$ to $\ints_5$,
so clearly $G/H \cong \ints_5$.

\section{Chapter 15, Exercise A4}

Denote the elements of $D_4$ as:

\begin{align}
	\label{topperm}
	R_0 = & \binom{1\ 2\ 3\ 4}{1\ 2\ 3\ 4} &
	R_{\pi/2} = & \binom{1\ 2\ 3\ 4}{4\ 1\ 2\ 3} &
	R_{\pi} = & \binom{1\ 2\ 3\ 4}{3\ 4\ 1\ 2} &
	R_{3\pi/2} = & \binom{1\ 2\ 3\ 4}{2\ 3\ 4\ 1} \\
	\label{botperm}
	H  = & \binom{1\ 2\ 3\ 4}{4\ 3\ 2\ 1} &
	V  = & \binom{1\ 2\ 3\ 4}{2\ 1\ 4\ 3} &
	D = & \binom{1\ 2\ 3\ 4}{3\ 2\ 1\ 4} &
	D' = & \binom{1\ 2\ 3\ 4}{1\ 4\ 3\ 2}
\end{align}

The operation table for
function composition $\circ$
on $D_4$
is given in table \ref{t1}

\begin{table}[!ht] 
\begin{tabular}{l|llllllll}
	$\circ$ & $R_0$ & $R_{\pi/2}$ & $R_\pi$ & $R_{3\pi/2}$ & $H$ & $V$ & $D$ & $D'$	\\ \hline
	$R_0$ & $R_0$ & $R_{\pi/2}$ & $R_\pi$ & $R_{3\pi/2}$ & $H$ & $V$ & $D$ & $D'$ \\
	$R_{\pi/2}$ & $R_{\pi/2}$ & $R_\pi$ & $R_{3\pi/2}$ & $R_0$ & $D'$ & $D$ & $H$ & $V$ \\
	$R_\pi$ & $R_\pi$ & $R_{3\pi/2}$ & $R_0$ & $R_{\pi/2}$ & $V$ & $H$ & $D'$ & $D$ \\
	$R_{3\pi/2}$ & $R_{3\pi/2}$ & $R_0$ & $R_{\pi/2}$ & $R_\pi$ & $D$ & $D'$ & $V$ & $H$ 	\\
	$H$ & $H$ & $D$ & $V$ & $D'$ & $R_0$ & $R_\pi$ & $R_{\pi/2}$ & $R_{3\pi/2}$ 	\\
	$V$ & $V$ & $D'$ & $H$ & $D$ & $R_\pi$ & $R_0$ & $R_{3\pi/2}$ & $R_{\pi/2}$ 	\\
	$D$ & $D$ & $V$ & $D'$ & $H$ & $R_{3\pi/2}$ & $R_{\pi/2}$ & $R_0$ & $R_\pi$	\\
	$D'$ & $D'$ & $H$ & $D$ & $V$ & $R_{\pi/2}$ & $R_{3\pi/2}$ & $R_\pi$ & $R_0$	\\
\end{tabular}
\centering
\caption{Operation table for $D_4$ under $\circ$}
\label{t2}
\end{table}

Now that we have better notation than Pinter,
let $G = D_4 \land H \le G \ni H = \{ R_0, R_\pi, H, V \}$

Note that other than $H$, $G/H$ contains only one other element since $(G:H) = 2$.

We can then fully describe $G/H = \Big\{ \{ R_0, R_\pi, H, V \}, \{R_{\pi/2}, R_{3\pi/2}, D, D' \} \Big\}$.

Table \ref{t3} give the operation table for $G/H$ under coset multiplication:

\begin{table}[!ht]
	\begin{tabular}{l|ll}
		$*$ & $H$ & $HD$ \\ \hline
		$H$ & $H$ & $HD$ \\
		$HD$ & $HD$ & $H$
	\end{tabular}
	\centering
	\caption{Operation table for $G/H$ under coset multiplication}
	\label{t3}
\end{table}

\section{Chapter 15, Exercise C1}

Suppose $H \trianglelefteq G$ where $G$ is a group.

\begin{thm} \label{thm1}
	$(\forall x \in G)(x^2 \in H) \iff (\forall X \in G/H)(X^2 = H)$
\end{thm}

\begin{proof}
	Suppose that $(\forall x \in G)(x^2 \in H)$.
	Let $X \in G/H$. Then, $X = Hx \ni x \in G$.
	Therefore, $XX = (Hx)(Hx) = H(x^2) = H$
	since $x^2 \in H \implies h(x^2) \in H$ for any $h \in H$
	since $H$ is closed under the group operation.
	Then, $(\forall X \in G/H)(X^2 = H)$.

	Conversely, suppose $(\forall X \in G/H)(X^2 = H)$.
	Let $x \in G$ and let $X = Hx$.
	Then, $X^2 = = (Hx)(Hx) = Hx^2$.
	By assumption, $X^2 = Hx^2 = H$,
	and by Pinter chapter 15 theorem 5 part 2,
	we have $Hx^2 = H \iff x^2 \in H$.

	The first implication and its converse thus proved
	demostrates bidirecitonal implication.
	This proves theorem \ref{thm1}.
\end{proof}

\section{Chapter 15, Exercise D1}

Suppose $H \trianglelefteq G$ where $G$ is a group.

\begin{thm} \label{thm2}
	$|H| \in \nats \land |G/H| \in \nats \implies |G| \in \nats$
\end{thm}

\begin{proof}
	Let $n = |H| \in \nats$ and let $m = |G/H|$.
	Since $G/H$ is the set of all left cosets of $H$ with respect to $G$,
	we can write $|G/H|$ as $(G:H)$, the index of $H$ with respect to $G$.
	By Lagrange's theorem, we have $|G| = (G:H) \cdot |H| = mn$.
	Since $m,n \in \nats$ and the naturals are closed under multiplication,
	we must have $|G| \in \nats$.
	This proves theorem \ref{thm2}.
\end{proof}

\section{Chapter 15, Exercise E2}

Suppose $H \trianglelefteq G$ where $G$ is a group.

\begin{thm} \label{thm3}
	$m = (G:H) \implies (\forall x \in G/H)( \ord(x) | m )$
\end{thm}

\begin{proof}
	Suppose $m = (G:H)$.
	Then, since $(G:H) = |G/H|$,
	we have as a consequence of
	Lagrange's theorem that
	$(\forall x \in G/H)(\ord(x)\ |\ |G/H|)$,
	therefore $(\forall x \in G/H)(\ord(x) | m)$.
	This proves theorem \ref{thm3}.
\end{proof}

\section{Chapter 15, Exercise E5}

Suppose $H \trianglelefteq G$ where $G$ is a group.

\begin{thm} \label{thm4}
	$m = (G:H) \implies a^m \in G$ for any $a \in G$
\end{thm}
\begin{proof}
	Suppose $m = (G:H)$ and
	let $a \in G$.
	Then, $a^m \in G$ since $G$ is
	closed under multiplication.
	This proves theorem \ref{thm4} for some reason.
\end{proof}

\section{Chapter 15, Exercise E6}

Suppose $H \trianglelefteq G$ where $G$ is a group.

\begin{thm} \label{thm5}
	$(\forall x \in \rats / \ints)(\ord(x) \in \nats)$
\end{thm}

\begin{proof}
	Suppose $x \in \rats / \ints$.
	Then $x$ can be written as
	some $y + \ints$,
	where $y \in \rats$.
	Then, we can write $y = \frac{m}{n}$
	for some $m,n \in \ints$
	by the definition of $\rats$.
	We want to find an $n \in \nats$
	such that $n(y + \ints) = \ints$.
	Any $n(y + \ints)$ is just $(ny + \ints)$,
	however this $n$ is no aribtrary integer,
	it is in fact the same $n$ we see in
	the denominator of $y = \frac{m}{n}$,
	for then $(ny + \ints) = (n\frac{m}{n} = \ints) = (m + \ints)$,
	and since $m \in \ints$, we have $m + \ints = \ints$,
	and every $x \in \rats / \ints$ satisfies $n(y + \ints) = \ints$
	for any $y = \frac{m}{n} \in \rats$,
	therefore $\ord(x) = n \in \nats$.
	This proves theorem \ref{thm5}.
\end{proof}


\section{The elements of factor group form an equivalence class}

Define the equivalence relation $\sim$
by $r \sim s \iff r - s \in \ints$ for some $r,s \in \rats$.
Let $[r]$ refer to the equivalence class of $r$ with respect to $\sim$.

\begin{thm} \label{thm6}
	$\rats / \ints = \{X: X = [r] \ni r \in \rats \}$.
\end{thm}

\begin{proof}
	Suppose $X \in \rats / \ints$.
	Then, $X$ is a coset of the form $r + \ints$
	where $r \in \rats$.
	Let $a,b \in X$.
	Then, $a = r + x_1 \ni x_1 \in \ints$
	and $b = r + x_2 \ni x_2 \in \ints$.
	Therefore, $a -b = (r + x_1) - (r + x_2) = x_1 - x_2 \in \ints \iff a \sim b \iff X = [r]$,
	therefore any $a,b \in X \implies a \sim b$,
	so any $X \in \rats / \ints \implies X = [r] \ni r \in \rats$.

	Conversely, suppose $X = [r] \ni r \in \rats$.
	Then, any $a,b \in X$ can be written as
	$r + x_1 \ni x_1 \in \ints$.
	This is exactly the definition of a coset of the integers under the rationals,
	so we must have $X \in \rats / \ints$.
	Then, $X \in \rats / \ints \implies X = [r] \ni r \in \rats \land
	X = [r] \ni r \in \rats \implies X \in \rats / \ints \iff
	\rats / \ints = \{ X: X = [r] \ni r \in \rats \}$
	by the axiom of extentionality.
	This proves theorem \ref{thm6}.
\end{proof}

\section{Factoring the alternating group of four elements}

Suppose $H = \{e, (12)(34), (13)(24), (14)(23) \}$ is a subgroup
of the $A_4 = \{e, (12)(34), (13)(24), (14)(23),(123),(132),(124),(142),(134),(143),(234),(243) \}$
alternating group of four elements.
In fact $H \trianglelefteq A_4$ by Homework 7 problem 12.

Then, we can write $A_4/H = \{H, H(123), H(132)\}$,
and table \ref{t4} gives the operation table for
$A_4/H$ under coset multiplication.

\begin{table}[!ht] 
\begin{tabular}{l|lll}
	$*$ & $H$ & $H(123)$ & $H(132)$ \\ \hline
	$H$ & $H$ & $H(123)$ & $H(132)$ \\
	$H(123)$ & $H(123)$ & $H(132)$ & $H$ \\
	$H(132)$ & $H(132)$ & $H$ & $H(123)$ \\
\end{tabular}
\centering
\caption{Operation table for $A_4/H$ under coset multiplication} 
\label{t4}
\end{table}

\begin{thm} \label{thm7}
	$Ha \in A_4/H \ni a \not\in H \implies \ord(Ha) = 3$.	
\end{thm}

\begin{proof}
	This is a simple proof by exhaustion.
	There are only two cases to check.
	First, observe that $(H(123))^2 = H(123)^2 = H(132)$,
	and $(H(123))^3 = H(132) \cdot H(123) = H((132)(123)) = H$,
	so $\ord(H(123)) = 3$.
	Then, observe that $(H(132))^2 = H(132)^2 = H(123)$,
	and $(H(132))^3 = H(123) \cdot H(132) = H((123)(132)) = H$,
	so $\ord(H(132)) = 3$.
	Since there are no other unique cosets where $a \not\in H$,
	we have shown that theorem \ref{thm7} holds for
	all possible cases, so this proves
	theorem \ref{thm7} in general.
\end{proof}

\section{Digging up an old equivalence relation}

Suppose $f,g \in \freals$.

Suppose we have
the function $\phi: \dreals \to \freals$
defined by $\phi(f) = \frac{df}{dx}$
where $x$ is the independent variable of $f$.
By the result of Homework 7 problem 2,
$\phi$ is an epimorphism from $\dreals$ to $\freals$,
so $\phi$ is a homomorphism from $\dreals$ to $\freals$.
Let $H = \ker(\phi)$ and define the relation $\sim$ as:

\begin{align} \label{sim}
	f \sim g \iff (\forall x \in \reals)(f(x) - g(x) = c \text{ for some } c \in \reals)
\end{align}

\begin{thm} \label{thm8}
	$(\forall f \in \dreals)(f + H = \{g \in \dreals: g = f + c \ni c \in \reals\})$
\end{thm}

\begin{proof}
	Suppose $f \in \dreals$.
	Then, $[f] = \{g \in \dreals: g \sim f \}$, and let $g \in [f]$.
	$f \sim g \iff (\forall x \in \reals)(f(x) - g(x) = c \ni c \in \reals)$.
	Since the neutral element of $\freals$ is $\epsilon(x) = 0$ and
	only a $c \in \reals$ satisfies $\frac{d}{dx} x = 0$,
	we can describe $H$ entirely by $H = \reals$,
	so clearly every $g \in [f]$ satisfies $g = f + c$
	and this is exactly the definition of $g$ being
	an element of the coset $f + H$,
	so any $g \in [f]$ satisfies $g \in f + H$.
	In fact, it works both ways, that any
	$g \in f + H$ can be written as some $g = f + c \ni c \in \reals$,
	so $g - f = c \in \reals$, implicitly, for any value of
	the independent variable of the two functions.
	This is exactly the definition that $f \sim g$,
	so we have $g \in [f]$,
	and by the axiom of extentionality
	we must have $f + H = \{g \in \dreals: g = f + c \ni c \in \reals\}$
	for any $f \in \dreals$. This proves theorem \ref{thm8}.
\end{proof}

\section{A homomorphism on continuous functions}

Suppose $G = \creals$ and define $\psi: G \to \reals \ni \psi(f) = \int_0^1 f(x)dx$.
We consider the group $G$ under function addition
and the group of the real numbers under conventional addition.

\begin{thm} \label{thm9}
	$\psi$ is a homomorphism with kernel $\ker(\psi) = H = \{f \in \creals: \int_0^1 f(x)dx = 0 \in \reals\}$.
\end{thm}

\begin{proof}
	Let $f,g \in \creals$.
	Then, $\psi(f + g) = \int_0^1 (f(x) + g(x)dx = \int_0^1 f(x)dx + \int_0^1 g(x)dx = \psi(f) \psi(g)$,
	so $\psi$ is a homomorphism.
	Since the additive identity of $\reals$ is $0$ and any for any $f \in \creals$,
	we have $\psi(f) = \int_0^1 f(x)dx$, we can fully describe the kernel of $\psi$
	by $\ker(\psi) = \{ f \in \creals : \int_0^1 f(x)dx = 0 \in \reals \}$.
	This proves theorem \ref{thm9}.
\end{proof}

By theorem \ref{thm9},
$G/H$ is defined since the kernel of a homomorphism
is a normal subgroup of the domain of that same homomorphism.
Then, we can descibe this quotient group
by $G/H = \{X: X = f + H \ni (\forall g \in H)(\int_0^1 (f(x) + g(x))dx = \int_0^1 g(x)dx) \}$.

\section{Normal subgroups of the general linear group in two dimensions}

Suppose $G = \gltwo$ and $H = \{ X \in G: \det(X) = 1 \} = \sltwo$ be a subgroup of $G$.

\begin{thm} \label{thm10}
	$H \trianglelefteq G$
\end{thm}

\begin{proof}
	Define the function $\phi: \gltwo \to \reals^* \ni \phi(A) = \det(A) \ni A \in \gltwo$.
	Then every $X \in \gltwo$ satisfies $\phi(X) = \det(X) = 1 \iff X \in \sltwo$
	because this is exactly the definition of an element being in $\sltwo$.
	Since for any $A,B \in \gltwo$, we have $\phi(AB) = \det(AB) = \det(A)\det(B) = \phi(A)\phi(B)$,
	we have that $\phi$ is a homomorphism from $\gltwo$ to $\reals^*$
	with $\ker(\phi) = \sltwo$.
	Then, since the kernel of a homomorphism is a normal subgroup of the domain,
	we have that $\sltwo$ is a normal subgroup of $\gltwo$,
	that is to say, $H \trianglelefteq G$ and this proves theorem \ref{thm10}.
\end{proof}

We can describe $G/H$
by $G/H =  \{X:X=Y\cdot\sltwo\ni Y\in \gltwo\}$.

\section{Another look at $\gltwo$}

Suppose $G = \gltwo$ and $H = \{X \in \gltwo: \det(X) > 0$.

Define $\phi:G \to P$ where $P = \{-1,1\}$ and for any $X \in \gltwo$,
we have $\phi(X) = \begin{cases} 1 & \textrm{ if } \det(X) > 0 \\
				-1 & \textrm{ if } \det(X) < 0 \end{cases}$.

Table \ref{t5} gives the operation table
for the parity group $P$
under multiplication.

\begin{table}[!ht]
	\begin{tabular}{l|ll}
		$*$ & $1$ & $-1$ \\ \hline
		$1$ & $1$ & $-1$ \\
		$-1$ & $-1$ & $1$
	\end{tabular}
	\centering
	\caption{Operation table for $P$ under mulitplication denoted by $*$}
	\label{t5}
\end{table}

\begin{thm} \label{thm11}
	$H \trianglelefteq G$
\end{thm}

\begin{proof}
	Supppose $X \in \gltwo$.
	By definition of $\gltwo$,
	$\det(X) \neq 0$,
	so we must have either
	$\det(X) > 0$
	or $\det(X) < 0$.

	We start with $X \in H \iff \det(X) > 0$,
	then $\phi(X) = 1$,
	and $X \in \ker(\phi)$.
	so $\det(X) > 0 \implies X \in \ker(\phi)$.

	Alternatively,
	if we start with $X \not\in H \iff \det(X) < 0$, 
	then $\phi(X) = -1$,
	and $X \not\in \ker(\phi)$
	so $\det(X) < 0 \implies X \not\in \ker(\phi)$.
	Since either $X \in H \lor X \not\in H$,
	we have that $\det(X) > 0 \iff X \in H$
	fully describes $\ker(\phi)$,
	so $X \in H \iff X \in \ker(\phi)$.

	Introduce another aribtrary $Y \in \gltwo$.
	Then, $\det(X)\det(Y) = \det(XY) > 0$
	so $\phi(XY) = \phi(X)\phi(Y)$,
	and $\phi$ is a homomorphism.


	Since $H$ is the kernel of a homomorphism
	from its group to some other group,
	$H$ must be a normal subgroup of $G$,
	i.e. $H \trianglelefteq G$.
	This proves theorem \ref{thm11}.
\end{proof}

Now we consider $G/H$.


\begin{thm} \label{thm12}
	$G/H = \{H, AH\}$ for some $A \in G \ni A \not\in H$.
\end{thm}

\begin{proof}
We first see that of course $e \in H \implies eH = H \in G/H$.
then, any $x \in H$ satisfies $aH = H$.
Suppose we have some $A \in G \not\in H \iff \det(A) < 0$.
We see that $AH \neq H$ since $A \not\in H$, so $AH \in G/H$.
Then, define $\psi:G/H \to P$ by $\psi(AH) = \phi(A)$ for any $A \in G$.
Since for any $AH, BH \in G/H$,
we have $\psi(AH)\psi(BH) = \phi(A)\phi(B) = \phi(AB) = \psi((AB))$,
$\psi$ is a homomorphism.

Suppose $\psi(AH) = \psi(BH)$.
Then, $\phi(A) = \phi(B)$,
so $\det(A) > 0 \iff \det(B) > 0$,
which means $A \in XH \iff B \in HX$
for any $X \in \gltwo$.
And since $A \in XH \iff AH = XH$
as well as $B \in XH \iff BH = XH$,
we have $AH = BH$,
so $\psi$ is injective.

Suppose $x \in P$.
Then, either $x = 1 \lor x = -1$.
Suppose $A \in G \ni \det(A) > 0$.
Then, $\psi(AH) = 1$.
Suppose $B \in G \ni \det(B) < 0$.
Then, $\psi(BH) = -1$.
For every element $x$ of $P$,
we can find an element $X$ of $G/H$
such that $\phi(X) = x$,
so by exhaustion,
$\psi$ is surijecive.

Since $\psi$ is injective and surjective,
$\psi$ is an isomorphism so its
domain and codomain
must have the same cardinality.
By inspection, we have $|P| = 2 \iff |G/H| = 2$.
Above, we saw that $H \in G/H$
and that $\exists A \in G$
where $AH \neq H$ holds,
so $\{H, AH\} \subseteq G/H$.
But since $|G/H| = 2$,
this must fully describe $G/H$
and $G/H \subseteq \{ H, AH \}$,
so $G/H = \{H, AH\}$
for some $A \in G \ni A \not\in H$
by the axiom of extentionality.
This proves theorem \ref{thm12}.
\end{proof}

\section{Quotient of a Group by Its Center}

Suppose $C \trianglelefteq G \ni C = \{x \in G : (\forall a \in G)(xa = ax) \}$.
Then $G/C$ is well-defined.

Assume that $G/C$ is cyclic, i.e.
we have that $\exists Ca \in G/C \ni \langle Ca \rangle = G/C$.

\begin{thm} \label{thm13}
	For any $x \in G$, there exists some $m \in \ints$ where $Cx = Ca^m$ holds.
\end{thm}

\begin{proof}
	Suppose $x \in G$.
	Then, $Cx \in G/C$ by definition.
	Since $G/C$ is cyclic,
	we have that any $Cx \in G/C$
	can be written as some $Cx = (Cb)^n \in G/C$,
	where $n \in \ints$.
	But since $(Cb)^n = Cb^n$,
	we have $Cx = Cb^n$,
	which proves theorem \ref{thm13}.
\end{proof}


\begin{thm} \label{thm14}
	For any $x \in G$, there exists some $m \in \ints$ where $x = ca^m \ni c \in C$ holds.
\end{thm}

\begin{proof}
	By theorem \ref{thm13},
	we have that $Cx = Ca^m$
	for some integer $m$,
	for any $x \in G$.
	Then, by the definition of a coset,
	we have some integers $c_1,c_2 \in C$,
	where $c_1x = c_2a^m$ holds,
	but multiplication on the left by $c_1^{-1}$
	yields $x = c_1^{-1}c_2a^m$,
	and since $C$ is a subgroup,
	we have some $c = c_1^{-1}c2 = \in C$,
	so $x = ca^m \ni c \in C$,
	which proves theorem \ref{thm14}.
\end{proof}

\begin{thm} \label{thm15}

	For any two $x,y \in G$, we have $xy = yx$.
\end{thm}

\begin{proof}
	By theorem \ref{thm14},
	any $x,y \in G$ can be written as
	$x = ca^m$ and $y = da^n$
	for some $c,d \in C$
	and some $m,b \in \ints$.
	Since by definition,
	any element of $C$ commutes
	with any element of $G$,
	we have $xy = (ca^m)(da^n) = c(da^m)a^n = (dc)a^{m + n}
	= (dc)a^{n + m} = d(ca^n)a^m = (da^n)(ca^m) = yx$,
	which proves theorem \ref{thm15}.
\end{proof}

\begin{thm} \label{thm16}
	If $G/C$ is cyclic, then $G$ is abelian.
\end{thm}

\begin{proof}
	Suppose $G/C$ were cyclic.
	Then, we have theorem \ref{thm12},
	and by theorem \ref{thm12},
	Then, we have theorem \ref{thm13},
	and by theorem \ref{thm13},
	we have theorem \ref{thm14},
	and by theorem \ref{thm14},
	we have theorem \ref{thm15},
	and by theorem \ref{thm15},
	we have that for any two
	$x,y \in G$, we have $xy = yx$,
	so $G$ is abelian.
	This proves theorem \ref{thm16}.
\end{proof}

\end{document}
