\documentclass[12pt]{article}
\usepackage{amsmath, amsthm, amsfonts, graphicx}

\usepackage{euler}
\title{Abstract Algebra: Homework \#7}
\author{Joel Savitz}
\date{Wednesday 8 July 2020}

\newcommand{\nats}{\mathbb{N}}
\newcommand{\reals}{\mathbb{R}}
\newcommand{\rats}{\mathbb{Q}}
\newcommand{\ints}{\mathbb{Z}}
\newcommand{\gltwo}{GL_2(\reals)}
\newcommand{\glmatrix}[4]{\ensuremath{\begin{bmatrix} #1 & #2 \\ #3 & #4 \end{bmatrix}}}
\newcommand{\glxmatrix}[9]{\ensuremath{\begin{bmatrix} #1 & #2 & #3 \\
#4 & #5 & #6 \\ #7 & #8 & #9 \end{bmatrix}}}
\newcommand{\glinverse}[4]{\ensuremath{\frac{1}{#1 #4 - #2 #3}\glmatrix{#4}{-#2}{-#3}{#1}}}
\newcommand{\ord}{\operatorname{ord}}
\newcommand{\Aut}{\operatorname{Aut}}
\newcommand{\freals}{\mathcal{F}(\reals)}
\newcommand{\dreals}{\mathcal{D}(\reals)}
\newtheorem{thm}{Theorem}
\newtheorem{cnt}{Counterexample}

\begin{document}
\maketitle

Note: I use the term ``epimorphism'' defined as a surjective homomorphism.
This kind of morphism is more relevant to the context of serveral of
the following proofs than mere homomorphism.
Every epimorphism is a homomorphism
so the proofs in question are only stonger
and should cover a superset of the relevant constraints.

\section{Chapter 14, Exercise A3}

Consider the surjection $f:\ints_{15} \to \ints_5$ defined as follows:

\begin{align*}
	f: 0 & \mapsto 0 &
	f: 1 & \mapsto 1 &
	f: 2 & \mapsto 2 &
	f: 3 & \mapsto 3 &
	f: 4 & \mapsto 4 \\
	f: 5 & \mapsto 0 &
	f: 6 & \mapsto 1 &
	f: 7 & \mapsto 2 &
	f: 8 & \mapsto 3 &
	f: 9 & \mapsto 4 \\
	f: 10 & \mapsto 0 &
	f: 11 & \mapsto 1 &
	f: 12 & \mapsto 2 &
	f: 13 & \mapsto 3 &
	f: 14 & \mapsto 4
\end{align*}

By inspection,
we see that $\ker(f) = \{0, 5, 10\}$.

\section{Chapter 14, Exercise B2}

Suppose we have
the function $\phi: \dreals \to \freals$
defined by $\phi(f) = \frac{df}{dx}$
where $x$ is the independent variable of $f$.

\begin{thm} \label{thm1}
	$\phi$ is an epimomorphism from $\dreals$ to $\freals$
\end{thm}

\begin{proof}
	Let $h' \in \freals$ be some arbitrary
	function where the prime is purely formal
	notation and does not denote an operation.
	Since $h'$ is continuous on $\reals$,
	it is integrable on the entire interval $(-\infty, \infty)$.
	By the first part of
	the fundamental theorem of calculus,
	we have that $h'$ is the derivative
	of some $h$ with repsect to the limits of integration
	implied by the integrability of $h'$.
	Then, denoting the independent variable of $h$ by $x$,
	we have $\phi(h) = \frac{d}{dx}h = h'$
	so we see that $\phi$ is surjective.

	Now suppose $f,g \in \dreals$
	with an independent variable denoted by $x$.
	By the linearity of
	the differentiation operation,
	we have $\phi(f) + \phi(g) =
	\frac{d}{dx}f + \frac{d}{dx}g =
	\frac{d}{dx}(f + g) = \phi(f + g) \in \freals$.
	Since $\phi$ has this property,
	it is a homomorphism.
	Furthermore, since $\phi$ is surjectve,
	it is an epimorphism.
	This proves theorem \ref{thm1}.
\end{proof}

We can describe the kernel of $\phi$ as
$\ker(\phi) = \reals \subseteq \dreals$,
since any constant function has a real zero derivative everywhere.

\section{Chapter 14, Exercise B4}

Suppose we have
the function $f:\reals^* \to \reals^{pos}$
defined by $f(x) = |x|$.
Note that the group operation
for both the domain and codomain of $f$
is good old real multiplication.

\begin{thm} \label{thm2}
	$f$ is an epimorphism from $\reals^*$ to $\reals^{pos}$.
\end{thm}

\begin{proof}
	Suppose $x,y \in \reals^*.$
	Then we see$f(x) \cdot f(y) = |x| \cdot |y| = |x \cdot y| = f(xy)$,
	so $f$ is a homomorphism.
	Now, consider some $a \in \reals^{pos}$.
	At least one of $f(a) = a$ or $f(-a) = a$ hold since they both hold,
	so $f$ is surjective.
	Then, $f$ is a surjective homomorphism
	if and only if
	$f$ is an epimorphism.
	This proves theorem \ref{thm2}
\end{proof}

We can write the kernel of $f$ as
$\ker(f) = \{ -1, 1\}$ since $|-1| = |1| = 1$
and $(\forall x \in \reals^{pos})(1x = x1 = x)$.

\section{Chapter 14, Exercise C2}

Suppose $f:G \to H$ is a homomorphism
and $\ker(f) = K$.

\begin{thm} \label{thm3}
	$f$ is injective if and only if $K = \{ e_G \}$
\end{thm}

\begin{proof}
	Suppose $f$ is injective.
	Then, any $x,y \in G$
	satisfy $f(x) = f(y) \implies x = y$.
	By definition of homomorphism, $f(e_G) = e_H$,
	so at the very least, $e_G \in K$.
	Suppose some other $x \in G$ satisfies $f(x) = e$.
	Then, $x = e_G$ by the reasoning previously elucidated.
	Therefore the cardinality of $K$ is $1$ and 
	$\ker(f)$ is the singleton $K = \{ e_G \}$.

	Conversely, suppose that $f$ is not injective.
	This means we must have some $x,y \in G$ where
	$f(x) = f(y) \implies x = y$ does not hold.
	
	Suppose $x,y \in G$ where $x \neq y$ satisfies $f(x) = f(y)$.
	Then, $f(x)[f(y)]^{-1} = e_H$,
	so $f(x)f(y^{-1}) = f(xy^{-1}) = e_H \iff xy^{-1} \in \ker(f)$.
	We have yet to confirm that the cardinality of $K$ exceeds
	the singleton $K$ seen above. To determine this property,
	consider the proposition $xy^{-1} = e_G$ as an assmption.
	We have $xy^{-1} = e_G \iff x = y$ by multiplication
	on the right by $y$, but by assumption, $x \neq y$,
	so we reach a contradiction and conclude that our assumption
	must be false,
	and as such, by the law of the excluded middle,
	we have $xy^{-1} \neq e_G$,
	and as such $\ker(f) = K \neq \{ e_G \}$ by violation
	of the axiom of extentionality.

	Since both the implication that
	if $f$ is injective
	then $K = \{ e_G \}$ holds
	and its converse holds,
	we must have the bidirective implication
	specified by theorem \ref{thm3},
	which proves the theorem.
\end{proof}

\section{Chapter 14, Exercise C4}

Suppose $f: G \to H$ is a homomorphism
and $J$ is some subgroup of $H$.

\begin{thm} \label{thm4}
	$f^{-1}(J) = \{ x \in G: f(x) \in J \}$ is a subgroup of $G$
	and $\ker(f) \subseteq f^{-1}(J)$
\end{thm}

\begin{proof}
	Suppose some $x,y \in G$ such that $f(x) \in J \land f(y) \in J$.
	Then, $f(x)f(y) = f(xy)$ since $f$ is a homomorphism
	and $f(xy) \in J$ since the group operation is closed
	in any subgroup, therefore $xy \in f^{-1}(J)$ by definition.
	Furthermore, since $e_H \in J$ by definition of subgroup,
	we must have that any $x \in G$ where $f(x) = e_H \in J$
	satisfies $x \in f^{-1}(J)$ by definition,
	therefore $x \in \ker(f) \implies x \in f^{-1}(J)$,
	so $\ker(f) \subseteq f^{-1}(J)$ holds
	and this proves theorem \ref{thm4}.
\end{proof}

\section{Chapter 15, Exercise D1}

\textbf{(a) Consider $S_3$.}

We have the following normal subgroups of $S_3$:

$\{e\}$ and $\{e, (123), (132) \}$ and $S_3$,
the first and last are trivial and the second
is by the fact that it has index $2$ with respect to $S_3$.

\textbf{(b) Consider $D_4$.}

Denote the elements of $D_4$ as:

\begin{align}
	\label{topperm}
	e = R_0 = & \binom{1\ 2\ 3\ 4}{1\ 2\ 3\ 4} &
	R_{\pi/2} = & \binom{1\ 2\ 3\ 4}{4\ 1\ 2\ 3} &
	R_{\pi} = & \binom{1\ 2\ 3\ 4}{3\ 4\ 1\ 2} &
	R_{3\pi/2} = & \binom{1\ 2\ 3\ 4}{2\ 3\ 4\ 1} \\
	\label{botperm}
	H  = & \binom{1\ 2\ 3\ 4}{4\ 3\ 2\ 1} &
	V  = & \binom{1\ 2\ 3\ 4}{2\ 1\ 4\ 3} &
	D = & \binom{1\ 2\ 3\ 4}{3\ 2\ 1\ 4} &
	D' = & \binom{1\ 2\ 3\ 4}{1\ 4\ 3\ 2}
\end{align}

We have the following normal subgroups of $D_4$:

\begin{itemize}
	\item $\{e\} = \{R_0\}$ 
	\item $\{R_0, R_\pi \}$ 
	\item $\{R_0, R_{\pi/2}, R_{\pi}, R_{3\pi/2} \}$ 
	\item $\{R_0, R_\pi, V, H\}$
	\item $\{R_0, R_\pi, D, D' \}$
	\item $D_4$
\end{itemize}

The first and last are trivial,
the second is by the fact that it is abelian,
and the rest are by the fact of the subgroups being of index $2$
with respect to $D_4$

\section{Chapter 14, Exercise D3}

\begin{thm} \label{thm5}
	The center of any group $G$ is a normal subgroup of $G$,
	i.e. $Z(G) \trianglelefteq G$
\end{thm}

\begin{proof}
	By the result of Homework 2 problem 12,
	we have that $Z(G)$ is a subgroup of $G$.
	Then, since $Z(G)$ satisfies $(\forall x\in Z(G))(\forall y \in G)(xy = yx)$,
	and by multiplication on the right by $y^{-1}$,
	we see that $x = yxy^{-1} \in Z(G)$ for any $y \in G$,
	so $Z(G)$ is a normal subgroup of $G$.
	This proves theorem \ref{thm5}.
\end{proof}

\section{Abelian group endomorphism}

\begin{thm} \label{thm6}
	$G$ is an Abelian group
	if and only if
	$f:G \to G \ni f(x) = x^{-1}$ is a homomorphism.
\end{thm}

\begin{proof}
	Suppose $G$ is an Abelian group.
	Then, any $x,y \in G$ satisfies $xy = yx$,
	so we must have $f(xy) = f(yx) = (xy)^{-1} = (yx)^{-1} = x^{-1}y^{-1} = f(x)f(y)$,
	so $G$ is a homomorphism.

	Conversely, suppose $f$ is a homomorphism.
	Then, any $x,y \in G$ satisfies $f(x)f(y) = f(xy)$,
	so we must have $f(x)f(y) = x^{-1}y^{-1} = (yx)^{-1} = y^{-1}x^{-1} = (xy)^{-1} = f(xy)$,
	and of course $(yx)^{-1} = (xy)^{-1} \iff yx = xy$.

	Then, $G$ is an abelian group
	if and only if
	$f$ is a homomorphism.
	This proves theorem \ref{thm6}.
\end{proof}

Since $f$ is a homomorphism from $G$ to itself,
$f$ is an endomorphism.

\section{Homomorphism via determinant}

Define $\phi: \gltwo \to \reals^* \ni \phi(x) = \det(x)$.

\begin{thm} \label{thm7}
	$\phi$ is a homomorphism
\end{thm}

\begin{proof}
	Since any two $A,B \in \gltwo$ satisfy
	$\phi(x)\phi(y) = \det(x)\det(y) = \det(xy) = \phi(xy)$
	by the properties of the determinant of a matrix,
	we have that $\phi$ is a homomorphism.
	This proves theorem \ref{thm7}.
\end{proof}

We can describe the kernel of $\phi$ by
$\ker(\phi) = \{ x \in \gltwo: \det(x) = 1 \}$,
otherwise known as the special linear group commonly denoted $SL_2(\reals)$.

\section{A normal subgroup of $\gltwo$}

\begin{thm} \label{thm8}
	$H = \{ X \in \gltwo: \det(X) > 0 \}$ is a normal subgroup of $\gltwo$,
	i.e. $H \trianglelefteq \gltwo$.
\end{thm}

\begin{proof}
	Suppose $A,B \in H$.
	Consider that $\det(AB) = \det(A)\det(B) > 0$
	as well as $\det(B)\det(A) = \det(BA) > 0$,
	since both $\det(A)$ and $\det(B)$ are positive by definition of $H$.
	Then, $AB \in H \iff BA \in H$ since the original choice of matrices was arbitrary,
	and $H$ is closed under matrix multiplication.
	Since $\det(A^{-1}) = \det(A)^{-1}$, we have $\det(A) > 0 \iff \det(A^{-1}) > 0$
	so every matrix $A$ in $H$ has its inverse $A^{-1}$ in $H$ and
	we see that $H$ is a subgroup of $G$.
	Finally, we have $\Big((H < G) \land (AB \in H \iff BA \in H)\Big) \iff H \trianglelefteq \gltwo$.
	This proves theorem \ref{thm8}.
\end{proof}

\section{Another normal subgroup of $\gltwo$}

\begin{thm} \label{thm9}
	$H = \{ X \in \gltwo: X = xI \ni x \in \reals^*\}$ is a normal subgroup of $\gltwo$,
	i.e. $H \trianglelefteq \gltwo$.
\end{thm}

\begin{proof}
	Suppose we have some $X \in H$ where $X = xI \ni x \in \reals^*$.
	Then consider any $A \in \gltwo$.
	Since any scalar commutes with any matrix,
	we have $AXA^{-1} = AxIA^{-1} = xAIA^{-1} = xAA^{-1} = xI = X \in H$,
	so any $X \in H$ has it's congugate with any $A \in G$ in the set
	$H$ since the conjugate is simply $X$ itself,
	thus $(X \in H \land A \in \gltwo \implies AXA^{-1} \in H) \iff H \trianglelefteq \gltwo$.
	This proves theorem \ref{thm9}.
\end{proof}

\section{A demonstration of $A_4 \trianglelefteq S_4$}

\begin{thm} \label{thm10}
	$A_4 \trianglelefteq S_4$
\end{thm}

\begin{proof}
	Since every permutation in $A_4$ is even by definition
	and an odd permutation composed with an even permuation is odd,
	the coset of $A_4$ generated by any odd permutation on $S_4$
	contains only odd permutations.
	Since every every permutation is either even or odd,
	we can comfortably conclude that there is only one coset of $A_4$ other than $A_4$ itself.
	Then, we see that $A_4$ has index $2$,
	so by the result of Homework 6 problem 14,
	$A_4$ must be a normal subgroup of $S_4$, i.e. $A_4 \trianglelefteq S_4$.
	This proves theorem \ref{thm10}.
\end{proof}

\section{A normal and a non normal subgroup}

Suppose $K = \{ e, (12)(34) \}$ and $H = \{e, (12)(34), (13)(24), (14)(23) \}$
are subgroups of $S_4$.

Let $a \not\trianglelefteq b$ denote the proposition $\neg(a \trianglelefteq b)$.

\begin{thm} \label{thm11}
	$K \trianglelefteq H \land K \not\trianglelefteq S_4$
\end{thm}

\begin{proof}
	By the counterexample $(13) \circ (12)(32) \circ (13) = (14)(23) \not\in K$,
	we clearly must have $K \not\trianglelefteq S_4$.
	Since every element of $K$ is in $H$, we have $K \subseteq H$,
	and since $(12)(34) \circ (12)(34) = e$ and $ee = e$,
	every element in $K$ has its inverse in $K$ and $K$ is closed
	under the group operation, so $K$ is a subgroup of $H$.
	Furthermore, $K$ is clearly of index $2$ with respect to $H$
	since it has one non-trivial coset,
	therefore by the result of Homework 6 problem 14,
	$K$ is a normal subgroup of $H$,
	therefore we have
	$K \trianglelefteq H \land K \not\trianglelefteq S_4$.
	This proves theorem \ref{thm11}.
\end{proof}

\section{Amazing Automorphisms And Analysis}

The automorphism of a group $G$ is the set of all isomorphisms from $G$ to itself,
foramlly: $\Aut(G) = \{ f \in S_G: f(ab) = f(a)f(b) \}$.

\medskip
\noindent
\textbf{(a) Conjugation is a homomorphism}

Let $G$ be a group.
Define $\pi_a: G \to G$ as $\pi_a(x) = axa^{-1}$ for some $a \in G$.

\begin{thm} \label{thm12}
	$\phi: G \to \Aut(G) \ni \phi(a) = \pi_a$ is a homomorphism.
\end{thm}

\begin{proof}
	Suppose $a,b \in G$.
	Then, consider $f(a)f(b) = \pi_a \circ \pi_b$.
	For an arbitrary $x \in G$,
	we see that $(\pi_a \circ \pi_b)(x) = a(bxb^{-1})a^{-1} = (ab)x(ab)^{-1}$.
	Then, since $f(ab) = \pi_{ab}$, we have $\pi_{ab}(x) = (ab)x(ab)^{-1}$.
	Thus, $f(ab) = f(a)f(b)$ so $f$ is a homomorphism.
\end{proof}

\medskip
\noindent
\textbf{(b) A normal subgroup of $\Aut(G)$}

Define $H = \{\pi_a \in \Aut(G): \pi_a(x) = ax^{-1} \ni a \in G \}$

\begin{thm} \label{thm13}
	$H \trianglelefteq \Aut(G)$
\end{thm}

\begin{proof}
	Suppose $\pi_a, \pi_b \in H$ and suppose $f \in \Aut(G)$.
	As demonstrated in the above proof, $\pi_a \circ \pi_b = \pi_{ab} \in H$,
	so $H$ is closed under the group operation.
	Of course, we can always find a $\pi_{a^{-1}} \in H$
	such that $(\pi_a \circ \pi_{a^{-1}})(x) = aa^{-1}x(aa^{-1})^{-1} = x$,
	so every element in $H$ has its inverse in $H$
	and we see that $H$ is a subgroup of $\Aut(G)$.

	Consider that $\pi_a(x) = axa^{-1}$ is simply the composition
	of permutations on $G$ denoted $a,x$ and $a^{-1}$ ---
	we can consider $x$ a permutation since it is in fact $\epsilon \in \Aut(G)$ ---
	and that $f$ and its inverse $f^{-1}$ are permutations as well
	since they are bijections from a set to itself.
	Then, we can multiply our first identity
	on the left by $f$
	and on the right by $f^{-1}$
	to get $f\pi_a(x)f^{-1} = f(axa^{-1})f^{-1} = (fa)x(fa)^{-1} = \pi_{fa} \in H$
	since $(fa)$ is some bijection from $G$ to itself, an element of $\Aut(G)$.
	Then, $\Big(z \in H \land f \in \Aut(G) \implies (fzf^{-1}) \in H\Big) \iff H \trianglelefteq \Aut(G)$.
	This proves theorem \ref{thm13}.
\end{proof}

\medskip
\noindent
\textbf{(c) The kernel of $\phi$}

\begin{thm} \label{thm14}
	$\ker(\phi) = \{ a \in G: ax = xa \ni a \in G \}$
\end{thm}

Denote the identity of $\Aut(G)$ by $\epsilon$
\begin{proof}
	Let $x \in G$ and consider the function $\epsilon$.
	We have $\epsilon(x) = x$ by definition.
	Since $\phi(a) = \pi_a$ for some $a \in G$,
	we can only satisfy $\pi_a(x) = \epsilon(x) = x$ when $axa^{-1} = x$.
	Then multiplication of this last identity
	by $a$ on the right yields $ax = xa$.
	We can generalize that for any $a \in G$, we have $\phi(a) = \epsilon \implies ax = xa$.
	In fact, we see that if we assume instead for any $a \in G \land x \in G$
	that $ax = xa$, we see that $x = axa^{-1} = \pi_a(x) = \epsilon(x) = \phi(a)(x)$,
	so we have the bidirective implication $\phi(a) = \epsilon \iff ax = xa$ for any $x \in G$.
	We can rewrite this fact as $\ker(\phi) = \{a \in G: ax = xa \ni a \in G \}$
	since this indeed satisfies the definition of kernel of a homomorphism.
	This proves theorem \ref{thm14}.
\end{proof}

\end{document}
