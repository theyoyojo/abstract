\documentclass[12pt]{article}
\usepackage{amsmath, amsthm, amsfonts, graphicx}

\usepackage{euler}
\title{Abstract Algebra: Homework \#6}
\author{Joel Savitz}
\date{Wednesday 1 July 2020}

\newcommand{\reals}{\mathbb{R}}
\newcommand{\rats}{\mathbb{Q}}
\newcommand{\ints}{\mathbb{Z}}
\newcommand{\gltwo}{GL_2(\reals)}
\newcommand{\glmatrix}[4]{\ensuremath{\begin{bmatrix} #1 & #2 \\ #3 & #4 \end{bmatrix}}}
\newcommand{\glxmatrix}[9]{\ensuremath{\begin{bmatrix} #1 & #2 & #3 \\
#4 & #5 & #6 \\ #7 & #8 & #9 \end{bmatrix}}}
\newcommand{\glinverse}[4]{\ensuremath{\frac{1}{#1 #4 - #2 #3}\glmatrix{#4}{-#2}{-#3}{#1}}}
\newcommand{\ord}{\operatorname{ord}}
\newcommand{\freals}{\mathcal{F}(\reals)}
\newtheorem{thm}{Theorem}
\newtheorem{cnt}{Counterexample}

\begin{document}
\maketitle

\section{Chapter 13, Exercise A3}

Suppose $H$ is a subgroup of some group $G$.
Furthermore, suppose
$G = \ints_{15} \land H = \langle 5 \rangle$.
Then,
denoting $+_{15}$ as $+$,
the following are the cosets of $H$:
\begin{align*}
	H + 0 = & \{0, 5, 10 \} \\
	H + 1 = & \{1, 6, 11 \} \\
	H + 2 = & \{2, 7, 12 \} \\
	H + 3 = & \{3, 8, 13 \} \\
	H + 4 = & \{4, 9, 14 \}
\end{align*}

\section{Chapter 13, Exercise A4}

Denote the elements of $D_4$ as:

\begin{align}
	\label{topperm}
	R_0 = & \binom{1\ 2\ 3\ 4}{1\ 2\ 3\ 4} &
	R_{\pi/2} = & \binom{1\ 2\ 3\ 4}{4\ 1\ 2\ 3} &
	R_{\pi} = & \binom{1\ 2\ 3\ 4}{4\ 2\ 1, 3} &
	R_{3\pi/2} = & \binom{1\ 2\ 3\ 4}{2\ 3\ 4\ 1} \\
	\label{botperm}
	H  = & \binom{1\ 2\ 3\ 4}{4\ 3\ 2, 1} &
	V  = & \binom{1\ 2\ 3\ 4}{2\ 1\ 4, 3} &
	D = & \binom{1\ 2\ 3\ 4}{3\ 2\ 1\ 4} &
	D' = & \binom{1\ 2\ 3\ 4}{1\ 4\ 3\ 2}
\end{align}

The operation table for
function composition $\circ$
on $D_4$
is given in table \ref{t1}

\begin{table}[!ht] 
\begin{tabular}{l|llllllll}
	$\circ$ & $R_0$ & $R_{\pi/2}$ & $R_\pi$ & $R_{3\pi/2}$ & $H$ & $V$ & $D$ & $D'$	\\ \hline
	$R_0$ & $R_0$ & $R_{\pi/2}$ & $R_\pi$ & $R_{3\pi/2}$ & $H$ & $V$ & $D$ & $D'$ \\
	$R_{\pi/2}$ & $R_{\pi/2}$ & $R_\pi$ & $R_{3\pi/2}$ & $R_0$ & $D'$ & $D$ & $H$ & $V$ \\
	$R_\pi$ & $R_\pi$ & $R_{3\pi/2}$ & $R_0$ & $R_{\pi/2}$ & $V$ & $H$ & $D'$ & $D$ \\
	$R_{3\pi/2}$ & $R_{3\pi/2}$ & $R_0$ & $R_{\pi/2}$ & $R_\pi$ & $D$ & $D'$ & $V$ & $H$ 	\\
	$H$ & $H$ & $D$ & $V$ & $D'$ & $R_0$ & $R_\pi$ & $R_{\pi/2}$ & $R_{3\pi/2}$ 	\\
	$V$ & $V$ & $D'$ & $H$ & $D$ & $R_\pi$ & $R_0$ & $R_{3\pi/2}$ & $R_{\pi/2}$ 	\\
	$D$ & $D$ & $V$ & $D'$ & $H$ & $R_{3\pi/2}$ & $R_{\pi/2}$ & $R_0$ & $R_\pi$	\\
	$D'$ & $D'$ & $H$ & $D$ & $V$ & $R_{\pi/2}$ & $R_{3\pi/2}$ & $R_\pi$ & $R_0$	\\
\end{tabular}
\centering
\caption{Operation table for $G$ under $\circ$}
\label{t1}
\end{table}

Suppose $H'$ is a subgroup of some group $G$.
Furthermore, suppose
$G = D_4 \land H' = \{R_0, D' \}$


Then,
using multiplicative notaiton for $\circ$,
we have the following cosets of $H'$
\begin{align*}
	H'R_0 = & \{ R_0, D' \} \\
	H'R_{\pi/2} = & \{ R_{\pi/2}, H \} \\
	H'R_\pi = & \{ R_\pi, D \} \\
	H'R_{3\pi/2} = & \{ R_{3\pi/2}, V \} \\
	H'H = & \{ H, R_{\pi/2} \} \\
	H'V = & \{ V, R_{3\pi/2} \} \\
	H'D = & \{ D, R_\pi \} \\
	H'D' = & \{ D', R_0 \} \\
\end{align*}

% \bibliographystyle{plain}
% \bibliography{document}

\section{Chapter 13, Exercise B1}

Suppose $H = \langle 3 \rangle$
where $3 \in \ints$.

Then, we can describe the three cosets of $H$ as follows:
\begin{align*}
	H + 0 = & \{x \in \ints:\exists k \in \ints \ni x = 3k \} \\
	H + 1 = & \{x \in \ints:\exists k \in \ints \ni x = 3k + 1\} \\
	H + 2 = & \{x \in \ints:\exists k \in \ints \ni x = 3k + 2\}
\end{align*}

\section{Chapter 13, Exercise C2}

Suppose $G$ is some group
such that $\ord(G) = pq$
for some prime natural $p$ and $q$.

\begin{thm} \label{thm1}
	$G$ is not cyclic if any only if every $x \in G \ni x \neq e \in G$
	satisfies $\ord(x) = p \lor \ord(x) = q$.
\end{thm}

\begin{proof}
	Suppose $G$ is cyclic.
	Then, some $x \in G$
	satisfies $\langle x \rangle = G \iff \ord(x) = pq$
	and we have some $x \in G$
	where $\ord(x) = p \lor \ord(x) = q$
	does not hold.

	Conversely,
	suppose $G$ is not cyclic.
	Then, let $x$ be some member of $G$
	where $x \neq e \in G$.
	By Lagrange's theorem,
	we must have that $\ord(x)$ divides $\ord(G)$,
	and so we have that $\ord(x)$ divides $pq$.
	Then, $\ord(x) \in \{1, p, q, pq\}$.
	We also have $\big(\ord(x) = 1 \iff x = e\big) \land \ord(x) \neq e\ \implies \ord(x) \neq 1$.
	And of course $\ord(x) \neq pq$,
	since otherwise $\langle x \rangle = G$,
	violating our assumption that $G$ is not cyclic.
	We have deduced that $\ord(x) = p \lor \ord(x) = q$ holds.

	This proves theorem \ref{thm1}.
\end{proof}

\section{Chapter 13, Exercise C3}

Suppose $G$ is some group
where $\ord(G) = 4$.

\begin{thm} \label{thm2}
	$G$ is not cyclic if and only if every element of $G$ is its own inverse.
\end{thm}

\begin{proof}
	Suppose $G$ is cyclic.
	Then, we have an $x \in G$
	such that $\langle x \rangle = G$.
	We can write $G$ as $\{e, x, x^2, x^3 \}$.
	By inspection we see that $x^2 \neq e$
	and we have an element of $G$
	that is not its own inverse,
	so it is false that every element of $G$
	is its own inverse when $G$ is cyclic.

	Suppose $G$ is not cyclic.
	By Lagrange's theorem,
	the order of every element of $G$
	must divide the order of $G$,
	so the non identity elements of $G$
	must have order $2$ or order $4$.
	Since $G$ is not cyclic,
	no element has order $4$,
	for if it did,
	that element would generate $G$
	and $G$ would not be cyclic.
	Since every $x \in G$
	satisfies $\ord(x) = 2$,
	we must have $x^2 = e$
	for every $x \in G$
	and then every element of $G$
	is its own inverse.

	This proves theorem \ref{thm3}.
\end{proof}

\begin{thm} \label{thm3}
	Every group of order $4$ is abelian.
\end{thm}

\begin{proof}
	Suppose $G$ is not cyclic.
	Then, by theorem \ref{thm2},
	we have that every $x \in G$
	satisfies $x^{-1} = x$.
	Applying this identity,
	we find that $ab = a^{-1}b^{-1} = (ba)^{-1} = ba$
	so any $a,b \in G$ commute
	and $G$ is abelian.

	Instead, suppose $G$ is cyclic.
	Then, $G$ has a generator $x$
	where $G = \{ e, x, x^2, x^3 \}$.
	Then, we can write any $y \in G$
	as $y = x^i$ for any $i \in \{0, 1, 2, 3 \}$.
	If $a = x^\alpha \in G$ and $b = x^\beta$
	are two such sets,
	we observe that $ab = x^\alpha x^\beta = x^{\alpha + \beta} = x^{\beta + \alpha} =
	x^\beta x^\alpha = ba$
	and see that any two $a,b \in G$
	commute and $G$ is abelian.

	Since $G$ is abelian if $G$ is cyclic
	and $G$ is abelian if $G$ is not cyclic,
	we see by the law of the exluded middle
	that $G$ is abelian and in general,
	every group of order $4$ is abelian.

	This proves theorem \ref{thm3}.
\end{proof}

\end{document}
