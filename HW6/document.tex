\documentclass[12pt]{article}
\usepackage{amsmath, amsthm, amsfonts, graphicx}

\usepackage{euler}
\title{Abstract Algebra: Homework \#6}
\author{Joel Savitz}
\date{Wednesday 1 July 2020}

\newcommand{\nats}{\mathbb{N}}
\newcommand{\reals}{\mathbb{R}}
\newcommand{\rats}{\mathbb{Q}}
\newcommand{\ints}{\mathbb{Z}}
\newcommand{\gltwo}{GL_2(\reals)}
\newcommand{\glmatrix}[4]{\ensuremath{\begin{bmatrix} #1 & #2 \\ #3 & #4 \end{bmatrix}}}
\newcommand{\glxmatrix}[9]{\ensuremath{\begin{bmatrix} #1 & #2 & #3 \\
#4 & #5 & #6 \\ #7 & #8 & #9 \end{bmatrix}}}
\newcommand{\glinverse}[4]{\ensuremath{\frac{1}{#1 #4 - #2 #3}\glmatrix{#4}{-#2}{-#3}{#1}}}
\newcommand{\ord}{\operatorname{ord}}
\newcommand{\freals}{\mathcal{F}(\reals)}
\newtheorem{thm}{Theorem}
\newtheorem{cnt}{Counterexample}

\begin{document}
\maketitle

\section{Chapter 13, Exercise A3}

Suppose $H$ is a subgroup of some group $G$.
Furthermore, suppose
$G = \ints_{15} \land H = \langle 5 \rangle$.
Then,
denoting $+_{15}$ as $+$,
the following are the cosets of $H$:
\begin{align*}
	H + 0 = & \{0, 5, 10 \} \\
	H + 1 = & \{1, 6, 11 \} \\
	H + 2 = & \{2, 7, 12 \} \\
	H + 3 = & \{3, 8, 13 \} \\
	H + 4 = & \{4, 9, 14 \}
\end{align*}

\section{Chapter 13, Exercise A4}

Denote the elements of $D_4$ as:

\begin{align}
	\label{topperm}
	R_0 = & \binom{1\ 2\ 3\ 4}{1\ 2\ 3\ 4} &
	R_{\pi/2} = & \binom{1\ 2\ 3\ 4}{4\ 1\ 2\ 3} &
	R_{\pi} = & \binom{1\ 2\ 3\ 4}{4\ 2\ 1, 3} &
	R_{3\pi/2} = & \binom{1\ 2\ 3\ 4}{2\ 3\ 4\ 1} \\
	\label{botperm}
	H  = & \binom{1\ 2\ 3\ 4}{4\ 3\ 2, 1} &
	V  = & \binom{1\ 2\ 3\ 4}{2\ 1\ 4, 3} &
	D = & \binom{1\ 2\ 3\ 4}{3\ 2\ 1\ 4} &
	D' = & \binom{1\ 2\ 3\ 4}{1\ 4\ 3\ 2}
\end{align}

The operation table for
function composition $\circ$
on $D_4$
is given in table \ref{t1}

\begin{table}[!ht] 
\begin{tabular}{l|llllllll}
	$\circ$ & $R_0$ & $R_{\pi/2}$ & $R_\pi$ & $R_{3\pi/2}$ & $H$ & $V$ & $D$ & $D'$	\\ \hline
	$R_0$ & $R_0$ & $R_{\pi/2}$ & $R_\pi$ & $R_{3\pi/2}$ & $H$ & $V$ & $D$ & $D'$ \\
	$R_{\pi/2}$ & $R_{\pi/2}$ & $R_\pi$ & $R_{3\pi/2}$ & $R_0$ & $D'$ & $D$ & $H$ & $V$ \\
	$R_\pi$ & $R_\pi$ & $R_{3\pi/2}$ & $R_0$ & $R_{\pi/2}$ & $V$ & $H$ & $D'$ & $D$ \\
	$R_{3\pi/2}$ & $R_{3\pi/2}$ & $R_0$ & $R_{\pi/2}$ & $R_\pi$ & $D$ & $D'$ & $V$ & $H$ 	\\
	$H$ & $H$ & $D$ & $V$ & $D'$ & $R_0$ & $R_\pi$ & $R_{\pi/2}$ & $R_{3\pi/2}$ 	\\
	$V$ & $V$ & $D'$ & $H$ & $D$ & $R_\pi$ & $R_0$ & $R_{3\pi/2}$ & $R_{\pi/2}$ 	\\
	$D$ & $D$ & $V$ & $D'$ & $H$ & $R_{3\pi/2}$ & $R_{\pi/2}$ & $R_0$ & $R_\pi$	\\
	$D'$ & $D'$ & $H$ & $D$ & $V$ & $R_{\pi/2}$ & $R_{3\pi/2}$ & $R_\pi$ & $R_0$	\\
\end{tabular}
\centering
\caption{Operation table for $G$ under $\circ$}
\label{t1}
\end{table}

Suppose $H'$ is a subgroup of some group $G$.
Furthermore, suppose
$G = D_4 \land H' = \{R_0, D' \}$


Then,
using multiplicative notaiton for $\circ$,
we have the following cosets of $H'$
\begin{align*}
	H'R_0 = & \{ R_0, D' \} \\
	H'R_{\pi/2} = & \{ R_{\pi/2}, H \} \\
	H'R_\pi = & \{ R_\pi, D \} \\
	H'R_{3\pi/2} = & \{ R_{3\pi/2}, V \} \\
	H'H = & \{ H, R_{\pi/2} \} \\
	H'V = & \{ V, R_{3\pi/2} \} \\
	H'D = & \{ D, R_\pi \} \\
	H'D' = & \{ D', R_0 \} \\
\end{align*}

% \bibliographystyle{plain}
% \bibliography{document}

\section{Chapter 13, Exercise B1}

Suppose $H = \langle 3 \rangle$
where $3 \in \ints$.

Then, we can describe the three cosets of $H$ as follows:
\begin{align*}
	H + 0 = & \{x \in \ints:\exists k \in \ints \ni x = 3k \} \\
	H + 1 = & \{x \in \ints:\exists k \in \ints \ni x = 3k + 1\} \\
	H + 2 = & \{x \in \ints:\exists k \in \ints \ni x = 3k + 2\}
\end{align*}

\section{Chapter 13, Exercise C2}

Suppose $G$ is some group
such that $\ord(G) = pq$
for some prime natural $p$ and $q$.

\begin{thm} \label{thm1}
	$G$ is not cyclic if any only if every $x \in G \ni x \neq e \in G$
	satisfies $\ord(x) = p \lor \ord(x) = q$.
\end{thm}

\begin{proof}
	Suppose $G$ is cyclic.
	Then, some $x \in G$
	satisfies $\langle x \rangle = G \iff \ord(x) = pq$
	and we have some $x \in G$
	where $\ord(x) = p \lor \ord(x) = q$
	does not hold.

	Conversely,
	suppose $G$ is not cyclic.
	Then, let $x$ be some member of $G$
	where $x \neq e \in G$.
	By Lagrange's theorem,
	we must have that $\ord(x)$ divides $\ord(G)$,
	and so we have that $\ord(x)$ divides $pq$.
	Then, $\ord(x) \in \{1, p, q, pq\}$.
	We also have $\big(\ord(x) = 1 \iff x = e\big) \land \ord(x) \neq e\ \implies \ord(x) \neq 1$.
	And of course $\ord(x) \neq pq$,
	since otherwise $\langle x \rangle = G$,
	violating our assumption that $G$ is not cyclic.
	We have deduced that $\ord(x) = p \lor \ord(x) = q$ holds.

	This proves theorem \ref{thm1}.
\end{proof}

\section{Chapter 13, Exercise C3}

Suppose $G$ is some group
where $\ord(G) = 4$.

\begin{thm} \label{thm2}
	$G$ is not cyclic if and only if every element of $G$ is its own inverse.
\end{thm}

\begin{proof}
	Suppose $G$ is cyclic.
	Then, we have an $x \in G$
	such that $\langle x \rangle = G$.
	We can write $G$ as $\{e, x, x^2, x^3 \}$.
	By inspection we see that $x^2 \neq e$
	and we have an element of $G$
	that is not its own inverse,
	so it is false that every element of $G$
	is its own inverse when $G$ is cyclic.

	Suppose $G$ is not cyclic.
	By Lagrange's theorem,
	the order of every element of $G$
	must divide the order of $G$,
	so the non identity elements of $G$
	must have order $2$ or order $4$.
	Since $G$ is not cyclic,
	no element has order $4$,
	for if it did,
	that element would generate $G$
	and $G$ would not be cyclic.
	Since every $x \in G$
	satisfies $\ord(x) = 2$,
	we must have $x^2 = e$
	for every $x \in G$
	and then every element of $G$
	is its own inverse.

	This proves theorem \ref{thm3}.
\end{proof}

\begin{thm} \label{thm3}
	Every group of order $4$ is abelian.
\end{thm}

\begin{proof}
	Suppose $G$ is not cyclic.
	Then, by theorem \ref{thm2},
	we have that every $x \in G$
	satisfies $x^{-1} = x$.
	Applying this identity,
	we find that $ab = a^{-1}b^{-1} = (ba)^{-1} = ba$
	so any $a,b \in G$ commute
	and $G$ is abelian.

	Instead, suppose $G$ is cyclic.
	Then, $G$ has a generator $x$
	where $G = \{ e, x, x^2, x^3 \}$.
	Then, we can write any $y \in G$
	as $y = x^i$ for any $i \in \{0, 1, 2, 3 \}$.
	If $a = x^\alpha \in G$ and $b = x^\beta$
	are two such sets,
	we observe that $ab = x^\alpha x^\beta = x^{\alpha + \beta} = x^{\beta + \alpha} =
	x^\beta x^\alpha = ba$
	and see that any two $a,b \in G$
	commute and $G$ is abelian.

	Since $G$ is abelian if $G$ is cyclic
	and $G$ is abelian if $G$ is not cyclic,
	we see by the law of the exluded middle
	that $G$ is abelian and in general,
	every group of order $4$ is abelian.

	This proves theorem \ref{thm3}.
\end{proof}

\section{Chapter 13, Exercise D1}

Suppose $H$ and $K$ are subgroups of a finite group $G$.

\begin{thm} \label{thm4}
	$H \subseteq K \implies (G:H) = (G:K)(K:H)$
\end{thm}

\begin{proof}
	Let $n = \ord(G)$
	and let $h = \ord(H) \land k = \ord(K)$.
	Then, by Lagrange's theorem,
	we must have that $h | n \land k | n$
	$(G:H) = \frac{\ord(G)}{\ord(H)}$ and
	$(G:K) = \frac{\ord(G)}{\ord(K)}$
	Since $H$ is a subgroup of $G$,
	we must have that
	$x \in H \implies x^{-1} \in H$
	and $(\forall x,y \in H)(xy \in H)$.
	Then, since we have $H \subseteq K$,
	we must have that $H$ is a subgroup of $K$,
	and therefore $(K:H) = \frac{\ord(K)}{\ord(H)}$
	By these identities, we must have:
	\begin{align}
		(G:H) = & \frac{\ord(G)}{\ord(H)} \\
		(G:H) = & \frac{\ord(G)\ord(K)}{\ord(H)\ord(K)} \\
		(G:H) = & \frac{\ord(G)}{\ord(K)} \frac{\ord(K)}{\ord(H)} \\
		(G:H) = & (G:K)(K:H)
	\end{align}
	This proves theorem \ref{thm4}.
\end{proof}

\section{Chapter 13, Exercise E1}

Suppose $H$ is a subgroup of some group $G$
and let $a,b \in G$.

\begin{thm} \label{thm5}
	$Ha = Hb \iff ab^{-1} \in H$
\end{thm}

\begin{proof}
	Suppose $Ha = Hb$.
	Then, we have $a \in Hb$
	so there is an $x \in H$ where $xb = a$,
	but then we can muliply both sides on the right by $b^{-1}$
	to see that $x = ab^{-1} \in H$.

	Conversely, suppose $ab^{-1} \in H$.
	Then, $a \in Hb$ since $(ab^{-1})b \in Hb$,
	but $a \in Hb \iff Ha = Hb$.
	
	This proves theorem \ref{thm5}.
\end{proof}

\section{Chapter 13, Exercise E3}

Suppose $H$ is a subgroup of some group $G$
and let $a,b \in G$.

\begin{thm} \label{thm6}
	$aH = Ha \land bH = Hb \implies (ab)H = H(ab)$
\end{thm}

\begin{proof}
	Suppose $aH = Ha \land bH = Hb$.
	If $x \in H$,
	then we have $xa = ax \land xb = bx$
	We can isolate the $x$ in each equation
	by multiplication of the first on the right by $a^{-1}$
	and multiplication of the second on the right by $b^{-1}$
	to get the identities $x = axa^{-1} \land x = bxb^{-1}$
	and substitute an $x$ in the first equation
	with an equivalent value in the second to get
	$x = a(bxb^{-1})a^{-1} = (ab)x(ab)^{-1}$.
	But then we can just multiply on the right by $(ab)$
	to get $x(ab) = (ab)x$
	and thus $x \in H(ab) \land x \in (ab)H \iff (ab)H = H(ab)$.
	This proves theorem \ref{thm6}.
\end{proof}

\section{The affine group and her little brother}

Suppose $G$ is the affine group
defined as $G = \Big\{ \glmatrix{a}{b}{0}{1} \in \gltwo: a \neq 0 \Big\}$.

Let $H = \Big\{ \glmatrix{1}{x}{0}{1}: x \in \reals \Big\}$.

\begin{thm} \label{thm7}
	$H$ is a subgroup of $G$
\end{thm}

\begin{proof}
	Suppose $x \in H$.
	We see that $x_{1,1} \in \reals \ni a \neq 0$
	and of course that $x_{1,2} \in \reals$,
	as well as the fact that $x_{2,1} = 0 \land x_{2,2} = 1$,
	so we conclude that $x \in G$
	and since $x \in H \implies x \in G$,
	we have $H \subseteq H$.

	Consider an $x = \glmatrix{1}{p}{0}{1} \in H$
	and a $y = \glmatrix{1}{q}{0}{1} \in H$
	Then, $xy = \glmatrix{1}{p + q}{0}{1} \in H$
	since $p + q \in \reals$
	and we see that $H$ is
	closed under matrix multiplication.

	Let $x = \glmatrix{1}{\alpha}{0}{1} \in H$.
	Then, $x^{-1} = \frac{1}{1\cdot1 - 0 \cdot \alpha}
	\glmatrix{1}{-\alpha}{0}{1} =
	\glmatrix{1}{-\alpha}{0}{1}$
	and clearly $x^{-1} \in H$ since $-\alpha \in \reals$.
	We also see that $xx^{-1} = x^{-1}x = \glmatrix{1}{0}{0}{1} \in H$.

	Then,
	since $H$ is
	a subset of $G$
	closed under matrix multiplication,
	where every element $x \in H$
	has its inverse $x^{-1} \in H$,
	we conclude that $H$ is a subgroup of $G$.
	This proves theorem \ref{thm7}.
\end{proof}

We can describe the right cosets of $H$ 
for some $a = \glmatrix{a}{b}{0}{1} \in G$
by $Ha = \Big\{k: k = \glmatrix{a}{b + x}{0}{1} \land k = yk \ni y = \glmatrix{1}{x}{0}{1} \in H \Big\}$.

\section{Cosets of some permutation group}

Suppose $H$ is a sugroup of $G = A_4$, where we can write $A_4$ as:
\begin{align} \label{a4}
	\{e, (12)(34), (13)(24),(14)(23),(123),(132),(124),(142),(134),(143),(234),(243)\}
\end{align}

and we let $H = \{ e, (12)(34), (13)(24),(14)(23) \}$.

We can calculate $(G:H) = \frac{\ord(G)}{\ord(H)} = \frac{12}{4} = 3$.

The three cosets of $H$ with respect to $G$ are:

\begin{align}
	He = &\ \{e, (12)(34), (13)(24),(14)(23) \} \\
	H(123) = &\ \{(123), (134), (243), (142) \} \\
	H(132) = &\ \{(132), (143),(234),(124) \}
\end{align}

\section{A bunch of proofs}

Suppose $B_1 = \{1, ..., k \}$ and $B_2 = \{k+1, ..., n\}$
where $k \in \ints \ni 1 \le k \le n -1$ for some $n \in \nats$.

Then, define the following two subgroups of $S_n$:
\begin{align}
	G_1 = &\ \{ f \in S_n: (\forall x \in B_1 \cup B_2)
	(x \in B_1 \implies f(x) \in B_1 \land x \in B_2 \implies f(x) = x) \} \\
	G_2 = &\ \{ f \in S_n: (\forall x \in B_1 \cup B_2)
	(x \in B_2 \implies f(x) \in B_2 \land x \in B_1 \implies f(x) = x) \}
\end{align}

Furthermore, define $H = \{ f \circ g: f \in G_1, g \in G_2 \}$.

\subsection{Elements of $G_1, G_2,$ and $H$, plus $(S_5:H)$}

Consider the concrete case of $S_5$.
Let $B_1 = \{1, 2\}$ and let $B_2 = \{3, 4, 5\}$.
Then, we can write the elements of $G_1, G_2,$ and $H$ as follows:

\begin{align*}
	G_1 = &\ \{e, (12)\} \\
	G_2 = &\ \{e, (34),(35),(45),(345),(354),\} \\
	H = &\ \{e, (34),(35),(45),(345),(354), \\
	    & (12), (12)(34),(12)(35),(12)(45),(12)(345),(12)(354) \}
\end{align*}

Since $\ord(S_5) = 5! = 120$
and $\ord(H) = 12$,
we have $(S_5:H) = \frac{120}{12} = 10$.

\subsection{Proof of $H \le S_n$ in general}

First, I need to prove general commutivity:

\begin{thm} \label{thm9}
	Any element of $G_1$ commutes with any element of $G_2$ under $\circ$	
\end{thm}

\begin{proof}
	Let $f \in G_1$ and let $g \in G_2$.
	Consider some $x \in B_1 \cup B_2$.
	We look at the possible values
	of $(f \circ g)(x) = f(g(x))$.
	If $x \in B_1$, then $g(x) = x$ and $f(g(x)) = f(x)$,
	but if $x \in B_2$, then $f(g(x)) = g(x)$.
	Alternatively,
	consider the possible values
	of $(g \circ f)(x) = g(f(x)$.
	If $x \in B_1$, then $g(f(x)) = f(x)$.
	but if $x \in B_2$, then $f(x) = x$ and $g(f(x)) = g(x)$.
	Since $\neg(x \in B_1) \iff (x \in B_2)$,
	we have that $(f \circ g)(x) = (g \circ f)(x)$
	for any $f \in G_1$ and $g \in G_2$.
	This proves theorem \ref{thm9}.
\end{proof}

Now, I can prove the following theorem:

\begin{thm} \label{thm8}
	$H$ is a subgroup of $S_n$
\end{thm}

\begin{proof}
	Let $x,y \in H$.
	By definition,
	we can write each $a \in H$
	as some $f \circ g \ni f \in G_1 \land g \in G_2$.
	As such, let $p \in G_1$ and $q \in G_2$ be
	such that $x = p \circ q$
	and let $r \in G_2$ and $s \in G_2$ be
	such that $y = r \circ s$.
	We can compose these to identities
	to get $x \circ y = (p \circ q) \circ (r \circ s)$.
	Then by theorem \ref{thm9} and the associativity of $\circ$,
	we have $x \circ y = (p \circ r) \circ (q \circ s)$,
	and since $(p \circ r) \in G_1$ and $(q \circ s) \in G_2$
	due to the closue of $\circ$ on subgroups $G_1$ and $G_2$,
	we have that $x \circ y$ is the composition of
	some element of $G_1$ and some element of $G_1$,
	and this is exactly the definition of $x \circ y \in H$.
	Then, $H$ is closed under $\circ$.

	If have $x = p \circ q \in H$,
	then we must have $x^{-1} = (p \circ q)^{-1} = (q^{-1} \circ p^{-1})$,
	and this is verified by $x^{-1} = (p \circ q) \circ (q^{-1} \circ q^{-1})$.
	Thus every $x \in H$ has its inverse $x^{-1} \in H$.

	With this last fact
	and with the fact that
	$H$ is closed under $\circ$,
	we conclude that
	$H$ is a subgroup of $S_n$
	and this proves theorem \ref{thm8}.
\end{proof}

\subsection{Abstract counting}

\begin{thm} \label{thm10}
	$(S_n:H) = \frac{n!}{k!(n-k)!}$
\end{thm}

\begin{proof}
	Since $G_1$ contains permutations
	on elements of $B_1$ only with all
	points in $B_2$ fixed and $|B_1| = k$,
	we have $\ord(G_1) = k!$.
	Then, since $G_2$ contains permutations
	on elememts of $B_2$ only with all
	points in $B_1$ fixed and $|B_2| = n - k$,
	we have $\ord(G_2) = (n - k)!$.
	Since we construct $H$
	by constraining the set to 
	some $k!$ elements of $G_1$
	composed with $(n-k)!$
	elements of $G_2$,
	where every composition is unique
	since they are on mutually
	exclusive intervalds of $\ints$,
	we have $\ord(G) = k!(n-k)!$.
	Finally because $\ord(S_n) = n!$,
	we must have by definition
	that $(S_n:H) = \frac{n!}{k!(n-k)!}$.
	This proves theorem \ref{thm10}.
\end{proof}

\section{A few equivalent propositions}

Suppose $a,b \in H$
where $H$ is a subgroup,
of some group $G$.

\begin{thm} \label{thm11}
	$a \in Hb \iff ab^{-1} \iff Ha = Hb$
\end{thm}

\begin{proof}
	By theorem \ref{thm5},
	we have $ab^{-1} \iff Ha = Hb$.
	Because $\Big(Ha = Hb \iff (x \in Ha \iff x \in Hb)\Big)$,
	we must have $Ha = Hb \iff a \in Hb$
	since clearly $a = ea \iff a \in Ha$.
	By transitivity and commutivity of bidirective implication,
	we have $a \in Hb \iff ab^{-1} \iff Ha = Hb$.
	This proves theorem \ref{thm11}.
\end{proof}

\section{Normal subgroups}

Define a normal subgroup of $G$ to be some $H$
such that $h \in H \land a \in G \implies aha^{-1} \in H$.

\begin{thm} \label{thm12}
	$\Big((\forall a \in H)(aH = Ha)\Big) \implies H \textrm{ is a normal subgroup of } G$.
\end{thm}

\begin{proof}
	Suppose that $aH = Ha$
	for any $a \in G$.
	Then, let $h$ be some
	element of $H$.
	Following our assumption,
	we must have $ha = ah$,
	which when each equivalent
	value is multiplied
	on the right
	by $a^{-1}$
	yields $h = aha^{-1} \in H$.
	Thus some $h \in H$ and any $a \in G$
	implies $aha^{-1} \in H$,
	so $H$ is a normal subgroup of $G$.
	This proves theorem \ref{thm12}.
\end{proof}

\section{Index $2$ subgroups are normal}

\begin{thm} \label{thm13}
	If $H$ is a subgroup of some $G$ where $(G:H) = 2$,
	then $H$ is a normal subgroup of $G$.
\end{thm}

\begin{proof}
	Suppose $H$ is a subgroup of some $G$
	where $(G:H) = 2$ holds.
	Let $h$ be some elemenet of $H$
	and let $a$ be some element of $G$.
	We have $a \in He = H$
	if and only if
	$Ha = He = H$
	by theorem \ref{thm11}.
	$a \in aH \iff Ha = aH$,
	and clearly $ae = a \in aH$,
	so $aH = Ha$ when $a \in H$.
	We have $a \not\in He \iff Ha \neq He = H$
	by theorem \ref{thm11},
	and then of course $a \not\in eH \iff aH \neq eH = H$.
	Since there are only two possible cosets of $H$
	by the fact that $(G:H) = 2$,
	and by the fact that cosets of $H$
	are disjoint partitions of the group $G$,
	we must have $aH \neq H \land Ha \neq H \iff aH = Ha$,
	so for any $a \in G$, we have $aH = Ha$.
	Then by theorem \ref{thm12},
	if we have that any $a \in G$
	satisfies $aH = Ha$,
	then $H$ is a normal subgroup of $G$.
	Since this is indeed the case with our generic
	subgroup  $H$ where $(G:H) = 2$,
	we must have $(G:H) = 2 \implies H \textrm{ is a normal subgroup of } G$.
	This proves theorem \ref{thm13}.
\end{proof}

\end{document}
