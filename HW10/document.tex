\documentclass[12pt]{article}
\usepackage{amsmath, amsthm, amsfonts, graphicx}

\usepackage{euler}
\title{Abstract Algebra: Homework \#10}
\author{Joel Savitz}
\date{Wednesday 29 July 2020}

\newcommand{\nats}{\mathbb{N}}
\newcommand{\reals}{\mathbb{R}}
\newcommand{\rats}{\mathbb{Q}}
\newcommand{\ints}{\mathbb{Z}}
\newcommand{\gltwo}{GL_2(\reals)}
\newcommand{\sltwo}{SL_2(\reals)}
\newcommand{\glmatrix}[4]{\ensuremath{\begin{bmatrix} #1 & #2 \\ #3 & #4 \end{bmatrix}}}
\newcommand{\glxmatrix}[9]{\ensuremath{\begin{bmatrix} #1 & #2 & #3 \\
#4 & #5 & #6 \\ #7 & #8 & #9 \end{bmatrix}}}
\newcommand{\glinverse}[4]{\ensuremath{\frac{1}{#1 #4 - #2 #3}\glmatrix{#4}{-#2}{-#3}{#1}}}
\newcommand{\ord}{\operatorname{ord}}
\newcommand{\Aut}{\operatorname{Aut}}
\newcommand{\freals}{\mathcal{F}(\reals)}
\newcommand{\dreals}{\mathcal{D}(\reals)}
\newcommand{\creals}{\mathcal{C}(\reals)}
\newcommand{\powset}{\mathcal{P}}
\newtheorem{thm}{Theorem}
\newtheorem{cnt}{Counterexample}

\begin{document}
\maketitle

\section{Chapter 17, Exercise B}

\textbf{1. Verification of ring properties}

Consider the group $\freals$ under function addition.

\begin{thm} \label{thm1}
	The set $\freals$ is a commmutative ring with unity under function addition and multiplication
\end{thm}

\begin{proof}
	Suppose $f,g,h \in \freals$.
	Since for any $x \in \reals$, we have $f + g = f(x) + g(x) = g(x) + f(x) = g + f$,
	the set $\freals$ forms an ableian group under function addition.
	We see that for any $x \in \reals$, it holds that $f(gh) = f(x)(g(x)h(x)) = (f(x)g(x))h(x) = (fg)h$,
	so multiplication is associative on the set $\freals$.
	Then, consider that for any $x \in \reals$,
	it is also the case that $f(g + h) = f(x)(g(x) + h(x)) = f(x)g(x) + f(x)h(x) = fg + fh$,
	and the same holds for multiplication of $g + h$ by $f$ on the right,
	therefore multiplication is distributive over additon.
	Since $\freals$ is an abelian group with associative multiplication that distributes over addition,
	$\freals$ is a ring, and since $fg = f(x)g(x) = g(x)f(x) = gf$, it is a commutative ring.

	Suppose $q \in \freals$ such that $\forall x \in \reals$ we have $q(x) = 0 \in \reals$.
	Then, $f + q = f(x) + q(x) = f(x) + 0 = f(x) = f$,
	so this $q$ must be the neutral element of $\freals$.

	Suppose $p \in \freals$ such that $\forall x \in \reals$ we have $p(x) = 1 \in \reals$.
	Then, $fp = f(x)p(x) = f(x)\cdot 1  = f(x) = f$
	so this $p$ must be the unity of $\freals$.

	Suppose $s \in \freals$ such that $\forall x \in \reals$ we have $s(x) = -f(x) \in \reals$.
	Then, $f + s = f(x) + s(x) = f(x) + -f(x) = 0$
	so this $s(x) = -f(x) $ must be negative of any $f \in \freals$.

	To conclude, we have shown that $\freals$ is a commutative ring with unity.
	This proves theorem \ref{thm1}.
\end{proof}

\noindent
\textbf{2. The Divisors of Zero}

Let $(h(x) = 0) \in \freals$ denote the zero function.

Suppose $f,g \in \freals$ where $f \neq h \neq g$ and $fg = h$

Since a real-valued function is the zero function if and only if
every element of the codomain maps to the real number zero,
any function that has nonzero value at some input is not the zero function,
however if for some $f$ and $g$, we have $f(x) = 0$ if and only if $g(x) \neq 0$
for any real $x$, then every $(fg)(x) = 0$.

Without loss of generality,
we can assume that $f$ has real nonzero value
on an interval $(-\infty,\alpha]$ for some $\alpha \in \reals$,
and $g$ has nonzero value on $(\alpha, \infty)$.
As established above, the constraint $f(x) = 0 \iff g(x) \neq 0$
implies that $fg = h \in \freals$.

More formally, we can describe the divisors of zero as follows:

\begin{align}
	f(x) = &
\begin{cases}
	q(x) \textrm{ if } x \le \alpha \\
	0 \textrm{ otherwise}
\end{cases} \\
	g(x) = &
\begin{cases}
	0 \textrm{ if } x \le \alpha \\
	q(x) \textrm{ otherwise}
\end{cases}
\end{align}

Where the arbitrary real-valued function $q$ is defined as:
$q: \reals \to \reals \ni (\forall x \in \reals)(q(x) \neq 0)$.
 
\noindent
\textbf{3. Invertible Elements}

Let $u \in \freals$ be the unity of the ring.

Some $f \in \freals$ is invertible
if and only if
there exists some $f^{-1} \in \freals$
such that for all $x \in \reals$,
we have $f(x)f^{-1}(x) = u(x)$.
We will denote this relationship by simply writing $ff^{-1} = u$,
or equivalently $f^{-1}f = u$ due to commutivity.

Since these functions are real-valued,
and any nonzero real number is invertible,
it is immediately obvious that a real-valued
function in this ring is invertible everywhere
is and only if it is not zero-valued at any input.

More formally:

\begin{align}
	(\forall f \in \freals)(\exists f^{-1} \in \freals \ni ff^{-1} = u \iff (\forall x \in \reals)(f(x) \neq 0))
\end{align}

\noindent
\textbf{4. Is this an integral domain or a field?}

The answer to both of these questions is no. Details follow.

A ring is an integral domain iff it has no divisors of zero.

The ring $\freals$ has at least one divisor of zero, therefore it is not an integral domain.

If a ring has some nonzero element that is not invertible, then it is not a field.

Define $f:\reals \to \reals \ni f(x) = \begin{cases} 0 \textrm{ if } x = 0 \\ 1 \textrm{ otherwise } \end{cases}$.

Clearly, $f$ is not invertible for reasons laid out in detail in part 3 of this answer.
Since there is some nonzero element in $\freals$ that is not invertible, we conclude that $\freals$ is not a field.

\section{Chapter 17, Exercise H3}

\begin{thm} \label{thm2}
	For any integral domain $A$, we have $a^2 = b^2 \implies a = \pm b$ for any $a,b \in A$.
\end{thm}

\begin{proof}
	Suppose $a,b \in A$ where $A$ is some integral domain and $a^2 = b^2$.
	Since we can write this as $aa = bb$, it is evident by inspection
	that if $a = b$, then $aa = bb$,
	but since $bb = (-b)(-b)$, we also have $aa = (-b)(-b)$,
	and so clearly if $a = -b$ then $aa = (-b)(-b) = bb$.
	Suppose we have some $c \in A \ni c \neq b \land c \neq -b$.
	Then, $c \neq b$ so $c^2 \neq b^2$,
	and since there are no other options,
	we conclude by exhaustion
	that $a = \pm b$.

	Alternatively we could just see from $a^2 = b^2 \iff a^2 - b^2 = (a - b)(a + b) = 0$
	that $a = \pm b$ since an integral domain has no divisors of zero.
	This proves theorem \ref{thm2}.
\end{proof}

\section{Chapter 17, Exercise H4}

\begin{thm} \label{thm3}
	For any integral domain $A$, we have $x \in A \ni x^2 = 1 \implies x = 1 \lor x = -1$
\end{thm}

\begin{proof}
	Suppose $x \in A \ni x^2 = 1$.
	Then, $x^2 - 1 = (x - 1)(x + 1) = 0$,
	and since there are no zero divisors
	in an integral domain,
	$x \in \{1, -1\}$.
	This proves theorem \ref{thm3}.
\end{proof}

\section{Chapter 17, Exercise I3}


\begin{thm} \label{thm4}
	For any nontrivial ring with unity $A$, $a^2 = 0 \implies
	\exists x,y \in A \ni x(a-1) = (a-1)x = 1 \land y(a+1) = (a+1)y = 1$.
\end{thm}

\begin{proof}
	Suppose $A$ is a nontrivial ring with unity.
	Let $a \in A \ni a^2 = 0$.
	Since $(a+1)(1-a) = (1-a)(a+1) = 1$, $a + 1$ is invertible,
	and since $(a - 1)(-(a+1)) = (-(a+1))(a-1) = 1$, $a - 1$ is invertible.
	Therefore, $a+1$ and $a-1$ are invertible and this proves theorem \ref{thm4}.
\end{proof}

\section{Chapter 17, Exercise J4}

\begin{thm} \label{thm5}
	For any nontrivial commutative ring $A$ and any two $a,b \in A \ni ab \neq 0$,
	we have $\exists x \in A \ni x \neq 0 \ni ax = 0 \lor bx = 0
	\implies \exists y \in A \ni yab = 0$.
\end{thm}

\begin{proof}
	Suppose $A$ is a nontrivial commutative ring
	and let $a,b \in A \ni ab \neq 0$.
	Then, without loss of generality suppose $a$ is a divisor of zero.
	Let $x \in A$ be some element where $xa = 0$.
	Then, $xab = 0b = 0$, so $ab$ is a divisor of zero.
	The same argument would apply if we chose $b$ initially,
	thus generality is maintained.
	This proves theorem \ref{thm5}.
\end{proof}

\section{Chapter 18, Exercise A1}

\begin{thm} \label{thm6}
	The set $A = \{x + \sqrt 3 y:x,y \in \ints\}$ is a subring of $\reals$.
\end{thm}

\begin{proof}
	Suppose $a,b \in A$.
	Then, we can express each of $a$ and $b$ as
	$a = x + \sqrt3y$ and $b = p + \sqrt3q$ where $x,y,p,q \in \ints$.
	Thus $a - b = (x - p) + (\sqrt 3)(y - q) \in A$,
	so $A$ is closed with respect to subtraction.
	In addition, $ab = (x + \sqrt 3 y)(p + \sqrt 3 q) = (xp + 3yq) + \sqrt 3 (yp + xq) \in A$,
	so $A$ is closed with respect to subtraction.
	A nonempty subset of some ring is closed with respect to
	subtraction and multiplication
	if and only if
	that nonempty subset is a subring of the ring,
	therefore $A$ is a subring of $\reals$.
	This proves theorem \ref{thm6}.
\end{proof}

\section{Chapter 18, Exercise B2}

The following are all of the ideals of $\ints_{12}$.
These are also the cyclic subgroups of the same set.

\begin{align*}
	\langle 0 \rangle & = \{0\} \text{ (trivial)} \\
	\langle 1 \rangle & = Z_{12} \text{ (trivial)} \\
	\langle 2 \rangle & = \{0, 2, 4, 6, 8, 10 \} \\
	\langle 3 \rangle & = \{0, 3, 6, 9 \} \\
	\langle 4 \rangle & = \{0, 4, 8 \} \\
	\langle 6 \rangle & = \{0, 6 \}
\end{align*}

\section{Chapter 18, Exercise B5}

Consider the subring $\creals$ of all continuous real functions of the ring $\freals$.

Let $f \in \freals$ be the step function,
where for any $x \in \reals$, $f(x) = y \in \ints \ni y < x \land
(\forall z \in \ints \ni z < y)(z \geq y \implies z = y)$, i.e, the floor of the input.

We see that $f \not\in \creals$ by the counterexample that
$\lim_{x \to 1^-} f(x) = 0 \neq \lim_{x \to 1^+} f(x) = 1$,
therefore
$\lim_{x \to 1} f(x)$ is undefined.

Then, let $g \in \creals \ni g(x) = 1$.

Since $fg$ still has a discontinuity at $x = 1$ (and every integer input),
we conclude that $\creals$ does not absorb products in $\freals$,
therefore the subring $\creals$ is not an ideal of the ring $\freals$.

\section{Chapter 18, Exercise D4}

\begin{thm} \label{thm7}
	For any ring $A$ and some $J \subseteq A \ni |J| > 0$,
	$\Big((\forall x,y \in J, z \in A)(x - y \in J \land xz \in J) \land 1 \in J \Big)\implies J = A$.
\end{thm}

\begin{proof}
	Suppose $a \in A$.
	Since $1 \in J$ and $(\forall x \in J, y \in A)(xy \in J)$,
	we have that $1a = a \in J$,
	therefore $A \subseteq J$.
	By assumption, $J \subseteq A$
	and by the axiom of extentionality, $A \subseteq J \land J \subseteq A \iff A = J$.
	This proves theorem \ref{thm7}.
\end{proof}

\section{Chapter 18, Exercise D5}

\begin{thm} \label{thm8}
	For any ring $A$ and some $J \subseteq A \ni |J| > 0$,
	$\Big((\forall x,y \in J, z \in A)(x - y \in J \land xz \in J)
	\land (\exists a \in J \ni \exists a^{-1} \in A)(aa^{-1} = aa^{-1} = 1) \Big)\implies J = A$.
\end{thm}

\begin{proof}
	Let $a^{-1} \in A$ be an element of $A$
	such that there exists an $a \in J$
	where $aa^{-1} = a^{-1}a = 1$.
	Since $(\forall x \in J, y \in A)(xy \in J)$,
	we have that $aa^{-1} = a^{-1}a = 1 \in J$,
	thereby satisfying the antecedent of theorem \ref{thm7}.
	Thus we conclude by theorem \ref{thm7} that $J = A$
	and this proves theorem \ref{thm8}.
\end{proof}

\section{Chapter 18, Exercise F1}

\begin{thm} \label{thm9}
	For any two rings $A$ and $B$ and a ring homomorphism $f:A \to B$,
	the set $f(A) = \{f(x): x \in A \}$ is a subring of $B$.
\end{thm}

\begin{proof}
	Let $a,b \in f(A)$.
	By the definition of $f(A)$,
	it is trivial that $f(A) \subseteq B$.
	By the same definition, we can rewrite $a$ and $b$
	as some $a = f(x)$ and $b = f(y)$ for some $x,y \in A$.
	Since $A$ is closed under subtraction,
	we have $f(x - y) \in A \land f(x - y) \in B$
	and by the properties of a homomorphism
	we have $f(x -y) = f(x) - f(y) \in B$,
	therefore $f(A)$ is closed under subtraction.
	Similarly, $f(xy) = f(x)f(y) \in A$,
	therefore $f(A)$ is also closed under multiplication.
	Since $f(A)$ is a subset of the ring $B$
	closed under subtraction and multiplication,
	$f(A)$ is a subring of the ring $B$.
	This proves theorem \ref{thm9}.
\end{proof}

\section{Chapter 18, Exercise F2}


\begin{thm} \label{thm10}
	For any two rings $A$ and $B$ and a ring homomorphism $f:A \to B$,
	we have that $(\forall x \in \ker(f), y \in A)(xy \in \ker(f))$.
\end{thm}

\begin{proof}
	Suppose $x \in \ker(f)$ and $y \in A$.
	Since $f$ is a homomorphism,
	we have $f(x)f(y) = f(xy)$.
	Since $x \in \ker(f)$, we have that $f(x) = 0$,
	therefore $f(x)f(y) = 0f(y) = 0 = f(xy)$,
	so $xy \in \ker(f)$ and we see that $\ker(f)$
	absorbs products in $A$.
	Now, let $z \in \ker(f)$.
	Since $f$ is a homomorphism, we have $f(x) - f(y) = 0 = f(x - y)$,
	so $x - y \in \ker(f)$ and $\ker(f)$ is closed under subtraction.
	Since $\ker(f)$ is closed under subtraction and absorbs products in $A$,
	we conclude that $\ker(f)$ is an ideal of $A$,
	or in other words,
	we have that $(\forall x \in \ker(f), y \in A)(xy \in \ker(f))$.
	This not only proves theorem \ref{thm10},
	this in fact proves the last theorem that I will be submitting
	for homework in this Abstract Abgebra class.
	Thank you Professor Lee for a great summer semester.
\end{proof}


\end{document}
